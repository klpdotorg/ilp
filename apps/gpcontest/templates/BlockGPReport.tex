\documentclass[12pt]{article}
\usepackage[a4paper,top=2cm,left=1cm,right=1cm,bottom=2cm,portrait]{geometry}
\usepackage{longtable}
\usepackage{graphicx}
\usepackage{makecell}
\usepackage{multirow}
\usepackage[table]{xcolor}
\usepackage{colortbl}
\usepackage{fancyhdr}
\usepackage[T1]{fontenc}
\usepackage{textcomp}
\usepackage{fontenc}
\usepackage{xltxtra}
\usepackage{tikz}
\usepackage{array}
\usepackage[absolute,overlay]{textpos}

\newcommand{\kn}[1]{
{\fontspec{Kedage}
#1
}}

\newcolumntype{P}[1]{>{\centering\arraybackslash}p{#1}}

\fancypagestyle{plain}{
\fancyhf{} % clear all header and footer fields
\lfoot{
\begin{tikzpicture}[overlay]
\node[anchor=south west,yshift=-1cm,xshift=-1cm]
{\includegraphics[height=2cm]{"{{info.imagesdir}}footer_left.png"}};
\end{tikzpicture}}

\rfoot{
\begin{tikzpicture}[overlay]
\node[anchor=south east,yshift=-1cm,xshift=1cm]
{\includegraphics[height=2cm]{"{{info.imagesdir}}footer_right.png"}};
\end{tikzpicture}}

\cfoot{
\fontsize{9}{12}
\begin{tikzpicture}[overlay]
\node[anchor=south west,yshift=0cm,xshift=-6cm]
{\kn{ಗ್ರಾಮ ಪಂಚಾಯ್ತಿ ಮಟ್ಟದ ಗಣಿತ ಸ್ಪರ್ಧೆಯ ವರದಿ} {{info.acadyear}} };
\node[anchor=south west,yshift=-0.5cm,xshift=-6cm]
{\kn{ಹೆಚ್ಚಿನ ಮಾಹಿತಿಗಾಗಿ} - 9845079590 \kn{ಗೆ ಕರೆ ಮಾಡಿ} };
\node[anchor=south,yshift=-0.5cm,xshift=6cm]
{\includegraphics[height=1cm]{"{{info.imagesdir}}footer_logo.png"}};
\end{tikzpicture}}
}

%% Headers and footers
\fancypagestyle{title}{
\fancyhf{} % clear all header and footer fields
\chead{
\begin{tikzpicture}[overlay]
\node[anchor=north, yshift=1.5cm, xshift=0cm]
{\includegraphics[width=3.5cm]{"{{info.imagesdir}}title_header_center.png"}};
\end{tikzpicture}}

\lhead{
\begin{tikzpicture}[overlay]
\node[anchor=north west,yshift=1.5cm,xshift=-1cm]
{\includegraphics[width=5cm]{"{{info.imagesdir}}title_header_left.png"}};
\end{tikzpicture}}

\rhead{
\begin{tikzpicture}[overlay]
\node[anchor=north east,yshift=1.5cm,xshift=1cm]
{\includegraphics[width=3cm]{"{{info.imagesdir}}title_header_right.png"}};
\end{tikzpicture}}

\begin{tikzpicture}[overlay]
\node[anchor=north west, yshift=-6.5cm, xshift=5cm]
{\includegraphics[width=7cm]{"{{info.imagesdir}}title_text.png"}};
\end{tikzpicture}

\begin{tikzpicture}[overlay]
\node[anchor=south west, yshift=-16.5cm,xshift=6cm]
{\includegraphics[width=4cm]{"{{info.imagesdir}}logo.jpg"}};
\end{tikzpicture}

\lfoot{
\begin{tikzpicture}[overlay]
\node[anchor=south west,yshift=0cm,xshift=-1cm]
{\includegraphics[height=4cm]{"{{info.imagesdir}}title_footer_left.png"}};
\end{tikzpicture}}

\rfoot{
\begin{tikzpicture}[overlay]
\node[anchor=south east,yshift=0cm,xshift=1cm]
{\includegraphics[height=3cm]{"{{info.imagesdir}}title_footer_right.png"}};
\end{tikzpicture}}

\setlength{\headheight}{15pt}
%%\setlength{\footskip}{120pt}
}


\renewcommand{\headrulewidth}{0pt} % to remove line on header
\renewcommand{\footrulewidth}{0pt} % to remove line on footer




\begin{document}

\begin{titlepage}
    \thispagestyle{title}
    \begin{center}
    \vspace*{10.5cm}
     \textbf{\huge \kn{ಗ್ರಾಮ ಪಂಚಾಯ್ತಿ ಮಟ್ಟದ}} \\~\\
        \textbf{\huge\kn{ಗಣಿತ ಸ್ಪರ್ಧೆಯ ವರದಿ}} \\~\\
	    \Large{{info.acadyear}}
    \end{center}
\end{titlepage}
\pagebreak

\thispagestyle{plain}
{
\setlength{\parindent}{0in}
\kn{ಇವರಿಗೆ,} \hfill  \kn{ {{info.month}} } {{info.year}} \\ [2ex]
\kn{ {{sendto.langname}} },\\ [1ex]
\kn{ {{blockinfo.block_langname}} } {{blockinfo.blockname}}\kn{ತಾಲೂಕು},\\ [1ex]
\kn{ {{blockinfo.district_langname}} } {{blockinfo.districtname}}\kn{ಜಿಲ್ಲೆ.
\\~\\ [3ex]
ಮಾನ್ಯರೆ,
\\~\\
\textbf{ವಿಷಯ:} ಕರ್ನಾಟಕ ಸರ್ಕಾರ, ಸಾರ್ವಜನಿಕ ಶಿಕ್ಷಣ ಇಲಾಖೆ ಹಾಗೂ ಅಕ್ಷರ ಫೌಂಡೇಶನ್ ಸಹಭಾಗಿತ್ವದಲ್ಲಿ ಗಣಿತ ಕಲಿಕಾ ಆಂದೋಲನದ ಅಡಿಯಲ್ಲಿ} 2019-2020\kn{ನೇ ಸಾಲಿನಲ್ಲಿ ನಡೆದ} \kn{ {{blockinfo.block_langname}} } {{blockinfo.blockname}} \kn{ತಾಲೂಕಿನ ಗ್ರಾಮ ಪಂಚಾಯತಿ ಮಟ್ಟದ ಸರ್ಕಾರಿ ಶಾಲಾ ಮಕ್ಕಳ ಗಣಿತ ಸ್ಪರ್ಧೆಯ ವರದಿ ಕುರಿತು.
\\~\\ [3ex]
ಗಣಿತ ಕಲಿಕಾ ಆಂದೋಲನವು ಸರ್ಕಾರದ ಕಾರ್ಯಕ್ರಮವಾಗಿದ್ದು, ಗ್ರಾಮ ಪಂಚಾಯತಿ, ಸಾರ್ವಜನಿಕ ಶಿಕ್ಷಣ ಇಲಾಖೆ, ಸಮುದಾಯ ಹಾಗೂ ಅಕ್ಷರ ಫೌಂಡೇಶನ್ ಸಹಯೋಗದೊಂದಿಗೆ} 2019-2020\kn{ನೆ ಸಾಲಿನಲ್ಲಿ} \kn{ {{blockinfo.block_langname}} } {{blockinfo.blockname}} \kn{ತಾಲೂಕಿನಲ್ಲಿ {{blockinfo.num_gps}} ಗ್ರಾಮ ಪಂಚಾಯತಿಯ, {{blockinfo.num_schools}} ಸರ್ಕಾರಿ ಶಾಲೆಗಳಿಂದ 4, 5 ಮತ್ತು 6ನೇ ತರಗತಿಯ ಒಟ್ಟು {{blockinfo.totalstudents}} ಮಕ್ಕಳಿಗೆ ಗಣಿತ ಸ್ಪರ್ಧೆಯನ್ನು ಮಾಡಲಾಗಿದೆ.}
\\~\\[2ex]
\kn{ಈ ತಾಲೂಕಿನ ಎಲ್ಲ ಗ್ರಾಮ ಪಂಚಾಯತಿಯ ಅಧ್ಯಕ್ಷರು/ಸದಸ್ಯರು, ಪೋಷಕರು, ಎಸ್.ಡಿ.ಎಮ್.ಸಿ ಅಧ್ಯಕ್ಷರು/ಸದಸ್ಯರು, ಸ್ವಯಂ ಸೇವಕರು, ಗ್ರಾಮದ ಮುಖಂಡರು, ಶಿಕ್ಷಣ ಪ್ರೇಮಿಗಳು, ಮುಖ್ಯಗುರುಗಳು/ ಶಿಕ್ಷಕರು ಮತ್ತು ಸ್ಥಳೀಯ ಸಂಘ ಸಂಸ್ಥೆಯ ಸದಸ್ಯರುಗಳು, ಸರ್ಕಾರಿ ಶಾಲಾ ಮಕ್ಕಳ ಗಣಿತ ಸ್ಪರ್ಧೆಗೆ ಸಂಪನ್ಮೂಲ, ಸಹಕಾರ ಮತ್ತು ಸಹಭಾಗಿತ್ವದೊಂದಿಗೆ ಭಾಗವಹಿಸಿ ಯಶಸ್ವಿಗೊಳಿಸಿದ್ದಾರೆ.
\\~\\[2ex]
ತಾಲೂಕಿನ ಗ್ರಾಮ ಪಂಚಾಯತಿಯ ಅಧ್ಯಕ್ಷರ ನಾಯಕತ್ವದಲ್ಲಿ, ಸರ್ಕಾರಿ ಶಾಲಾ ಮಕ್ಕಳ ಗುಣಾತ್ಮಕ ಶಿಕ್ಷಣಕ್ಕಾಗಿ ಗಣಿತ ಸ್ಪರ್ಧೆಗಳನ್ನು ನಿರಂತರವಾಗಿ ಪಂಚಾಯತಿ ಮಟ್ಟದಲ್ಲಿ ನಡೆಯುವಂತೆ ಆಗಬೇಕೆಂದು ಕೋರುತ್ತೇವೆ.
\\~\\
ಈ ಪತ್ರದೊಂದಿಗೆ ತಮ್ಮ ತಾಲೂಕಿನ ಎಲ್ಲ ಪಂಚಾಯತಿಗಳ ಗಣಿತ ಸ್ಪರ್ಧೆಯ ಕ್ರೂಢೀಕೃತ ವರದಿಯನ್ನು ಮುಂದಿನ ಯೋಜನೆ ಮತ್ತು ಸೂಕ್ತ ಕ್ರಮಕ್ಕಾಗಿ ಲಗತ್ತಿಸಲಾಗಿದೆ.
\\~\\~\\~\\~\\
ವಂದನೆಗಳೊಂದಿಗೆ
\\~\\~\\~\\
ಅಶೋಕ್ ಕಾಮತ್,
\\~\\
ಅಧ್ಯಕ್ಷರು ,ಅಕ್ಷರ ಫೌಂಡೇಶನ್\\[1ex]
ಬೆಂಗಳೂರು\\
}
}

\pagebreak



\thispagestyle{plain}
{
\setlength{\parindent}{0in}
\begin{center}\large{\textbf{\kn{ಗಣಿತ ಕಲಿಕಾ ಆಂದೋಲನ}}} \small{ {{info.acadyear}} } \\ \end{center}

~\\~\\

\kn{ಗ್ರಾಮ ಪಂಚಾಯತಿ ಮಟ್ಟದ ಶಾಲಾ ಮಕ್ಕಳ ಗಣಿತ ಸ್ಪರ್ಧೆಯ ಜಿಲ್ಲಾ ವರದಿ} \\
\kn{ತಾಲೂಕಿನ ಹೆಸರು} : \kn{ {{blockinfo.block_langname}} } {{blockinfo.blockname}}  \\ 
\kn{ಜಿಲ್ಲೆಯ ಹೆಸರು} : \kn{ {{blockinfo.district_langname}} } {{blockinfo.districtname}}\\
\kn{ಸ್ಪರ್ಧೆಯಲ್ಲಿ ಭಾಗವಹಿಸಿದ ಗ್ರಾಮ ಪಂಚಾಯ್ತಿಗಳ ಸಂಖ್ಯೆ} : {{blockinfo.num_gps}} \\

\kn{ {{blockinfo.block_langname}} }{{blockinfo.blockname}} \kn{ತಾಲೂಕಿನಲ್ಲಿ ಗ್ರಾಮ ಪಂಚಾಯತಿ, ಸಾರ್ವಜನಿಕ ಶಿಕ್ಷಣ ಇಲಾಖೆ, ಸಮುದಾಯ ಮತ್ತು ಅಕ್ಷರ ಫೌಂಡೇಶನ್ ಸಹಭಾಗಿತ್ವದಲ್ಲಿ ಗ್ರಾಮ ಪಂಚಾಯತಿ ಮಟ್ಟದಲ್ಲಿ ನಡೆದ 4,5 ಮತ್ತು 6 ನೇ ತರಗತಿಯ ಶಾಲಾ ಮಕ್ಕಳ ಗಣಿತ ಸ್ಪರ್ಧೆಯ ಕಿರು ವರದಿ ಈ ಕೆಳಕಂಡಂತಿದೆ}\\[2ex]
\kn{ಈ ತಾಲೂಕಿನ ವ್ಯಾಪ್ತಿಯಲ್ಲಿ} {{blockinfo.gp_count}} \kn{ಗ್ರಾಮ ಪಂಚಾಯತಿಗಳಲ್ಲಿ  ಗಣಿತ ಸ್ಪರ್ಧೆ ನಡೆದಿದ್ದು,} {{blockinfo.school_count}} \kn{ಸರ್ಕಾರಿ ಶಾಲೆಗಳಿಂದ 4,5 ಮತ್ತು 6 ನೇ ತರಗತಿಯ} {{blockinfo.totalstudents}} \kn{ಮಕ್ಕಳು ಸ್ಪರ್ಧೆಯಲ್ಲಿ ಭಾಗವಹಿಸಿದ್ದರು.}\\[2ex]

\textbf{\kn{ತರಗತಿವಾರು ಗಣಿತ ಸ್ಪರ್ಧೆಯಲ್ಲಿ ಭಾಗವಹಿಸಿದ್ದ ಮಕ್ಕಳ ಒಟ್ಟಾರೆ ಫಲಿತಾಂಶಗಳು ಈ ಕೆಳಗಿನಂತೆ ಇವೆ.}}

\begin{longtable}{|P{3cm}|P{4cm}|P{2cm}|P{2cm}|P{2cm}|P{3cm}|} \hline
\textbf{\kn{ತರಗತಿ}} & \textbf{\kn{ಭಾಗವಹಿಸಿದ ಮಕ್ಕಳ ಸಂಖ್ಯೆ}} & \multicolumn{4}{|c|}{\textbf{\kn{ಅಂಕ ಗಳಿಸಿದ ಮಕ್ಕಳ ಸಂಖ್ಯೆ (ಶೇಕಡವಾರು)}}}\\ \cline{3-6}
\rule{0cm}{0.3cm}& &  \textbf{<35\%} & \textbf{35\%-60\%}  & \textbf{61\%-75\%} & \textbf{76\%-100\%} \\ \hline \endhead

\rule{0cm}{0.3cm} {{ assessment["class"] }}\kn{ನೇ ತರಗತಿ} & {{ assessment["overall_scores"]["total"]}} & {{assessment["overall_scores"]["below35"]}} & {{assessment["overall_scores"]["35to60"]}} & {{assessment["overall_scores"]["60to75"]}} & {{assessment["overall_scores"]["75to100"]}} \\ \hline

\end{longtable}

\begingroup
\fontsize{11}{12}
{
\setlength\LTleft{-0.1cm}
\setlength\LTright{-0.1cm}
\begin{longtable}{|P{2.5cm}|P{3.25cm}|c|c|c|c|c|c|} \hline
\textbf{\kn{ತರಗತಿ}} & \textbf{\kn{ಭಾಗವಹಿಸಿದ ಮಕ್ಕಳ ಸಂಖ್ಯೆ}} & \multicolumn{6}{|c|}{\textbf{\kn{ಮೂಲ ಕ್ರಿಯೆಗಳನ್ನು ಸರಿಯಾಗಿ ಮಾಡಬಲ್ಲ ಮಕ್ಕಳು}}} \\ \cline{3-8}
\rule{0cm}{0.3cm} &&\textbf{\kn{ಸಂಖ್ಯಾ ಪರಿಚಯ}}&\textbf{\kn{ಸ್ಥಾನ ಬೆಲೆ}}&\textbf{\kn{ಸಂಕಲನ}}&\textbf{\kn{ವ್ಯವಕಲನ}}&\textbf{\kn{ಗುಣಾಕಾರ}}&\textbf{\kn{ಭಾಗಾಕಾರ}} \\ \hline \endhead

\rule{0cm}{0.3cm} {{assessment["class"]}}\kn{ನೇ ತರಗತಿ} & {{assessment["competency_scores"]["total"]}} & {{assessment["competency_scores"]["Number Recognition"]}}  & {{assessment["competency_scores"]["Place Value"]}} & {{assessment["competency_scores"]["Addition"]}} & {{assessment["competency_scores"]["Subtraction"]}} & {{assessment["competency_scores"]["Multiplication"]}} & {{assessment["competency_scores"]["Division"]}} \\ \hline

\end{longtable}
}
\endgroup
 

{
\centering{\kn{ಫಲಿತಾಂಶ : ಒಟ್ಟಾರೆಯಾಗಿ 5 ವರ್ಷ(6 ನೇ ತರಗತಿ) ಶಾಲೆಯಲ್ಲಿ ಕಲಿತ ಮಗು ಈ ಕೆಳಗಿನ ಮೂಲಕ್ರಿಯೆಯನ್ನು ಮಾಡಬಹುದು ಎಂದು ನಾವು ಪರಿಗಣಿಸಬಹುದು} (\kn{ಶೇಕಡವಾರು}).} \\[2ex]
 
\begingroup
\fontsize{8.75}{12}
{

{
\centering{
\begin{longtable}{|c|c|c|c|c|c|} \hline
\cellcolor{gray}{\textbf{\makecell[b]{\kn{ಸಂಖ್ಯಾ ಪರಿಚಯವನ್ನು} \\ \kn{ಸರಿಯಾಗಿ ಮಾಡಬಲ್ಲ ಮಕ್ಕಳು}}}} & \cellcolor{lightgray}{\textbf{\makecell[b]{\kn{ಸ್ಥಾನ ಬೆಲೆಯನ್ನು ಸರಿಯಾಗಿ}\\ \kn{ಮಾಡಬಲ್ಲ ಮಕ್ಕಳು}}}} & \textbf{\makecell[b]{\kn{ಸಂಕಲನವನ್ನು ಸರಿಯಾಗಿ}\\ \kn{ಮಾಡಬಲ್ಲ ಮಕ್ಕಳು}}} & \cellcolor{gray}{\textbf{\makecell[b]{\kn{ವ್ಯವಕಲನವನ್ನು ಸರಿಯಾಗಿ}\\ \kn{ಮಾಡಬಲ್ಲ ಮಕ್ಕಳು}}}} & \cellcolor{lightgray}{\textbf{\makecell[b]{\kn{ಗುಣಾಕಾರವನ್ನು ಸರಿಯಾಗಿ}\\ \kn{ಮಾಡಬಲ್ಲ ಮಕ್ಕಳು}}}} & \textbf{\makecell[b]{\kn{ಭಾಗಾಕಾರವನ್ನು ಸರಿಯಾಗಿ}\\ \kn{ಮಾಡಬಲ್ಲ ಮಕ್ಕಳು}}} \\ \hline \endhead

\rule{0cm}{0.3cm} {{percent_scores["assessments"][assessment]["Number Recognition"]}}&  {{percent_scores["assessments"][assessment]["Place Value"]}} & {{percent_scores["assessments"][assessment]["Addition"]}} & {{percent_scores["assessments"][assessment]["Subtraction"]}} & {{percent_scores["assessments"][assessment]["Multiplication"]}} &  {{percent_scores["assessments"][assessment]["Division"]}}  \\ \hline

\end{longtable}
}
}

{
\centering{
\begin{longtable}{|c|c|c|c|} \hline
 \textbf{\makecell[b]{\kn{ಸಂಕಲನವನ್ನು ಸರಿಯಾಗಿ}\\ \kn{ಮಾಡಬಲ್ಲ ಮಕ್ಕಳು}}} & \cellcolor{gray}{\textbf{\makecell[b]{\kn{ವ್ಯವಕಲನವನ್ನು ಸರಿಯಾಗಿ}\\ \kn{ಮಾಡಬಲ್ಲ ಮಕ್ಕಳು}}}} & \cellcolor{lightgray}{\textbf{\makecell[b]{\kn{ಗುಣಾಕಾರವನ್ನು ಸರಿಯಾಗಿ}\\ \kn{ಮಾಡಬಲ್ಲ ಮಕ್ಕಳು}}}} & \textbf{\makecell[b]{\kn{ಭಾಗಾಕಾರವನ್ನು ಸರಿಯಾಗಿ}\\ \kn{ಮಾಡಬಲ್ಲ ಮಕ್ಕಳು}}} \\ \hline \endhead

\rule{0cm}{0.3cm} {{percent_scores["assessments"][assessment]["Addition"]}} & {{percent_scores["assessments"][assessment]["Subtraction"]}} & {{percent_scores["assessments"][assessment]["Multiplication"]}} &  {{percent_scores["assessments"][assessment]["Division"]}}  \\ \hline
	
\end{longtable}
}
}

}
\endgroup
}


\begin{flushleft}
\kn{ಈ ಮೇಲ್ಕಂಡ ಫಲಿತಾಂಶಗಳು ಇನ್ನೂ ಉತ್ತಮ ಪಡಿಸಲು ಸಾರ್ವಜನಿಕ ಶಿಕ್ಷಣ ಇಲಾಖೆ ,ಗ್ರಾಮ ಪಂಚಾಯತಿ ಜೊತೆಗೆ ಇತರೆ ಭಾಗೀದಾರರಾದ ನಾಗರೀಕ ಸೌಕರ್ಯ ಸಮಿತಿ}(Civic Amenities Committee), SDMC, \kn{ಪೋಷಕರು, ಶೈಕ್ಷಣಿಕ ಸ್ವಯಂಸೇವಕರು ಮತ್ತು ಶಾಲಾ ಶಿಕ್ಷಕರ ಜೊತೆಗೂಡಿ ಯೋಜನೆಯನ್ನು ರೂಪಿಸುತ್ತೀರಾ ಎಂದು ಭಾವಿಸುತ್ತೇವೆ.} \\[5ex]
\end{flushleft}


\centering{\kn{ವಂದನೆಗಳೊಂದಿಗೆ}} \\[3ex]

~\\~\\~\\

\begin{flushleft}
\kn{ಇಂದ,\\
ಅಕ್ಷರಫೌಂಡೇಶನ್}
\end{flushleft}
\end{document}
