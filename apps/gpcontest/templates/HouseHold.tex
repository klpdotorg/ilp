\documentclass[12pt]{article}
\usepackage[a4paper,top=2cm,left=1cm,right=1cm,bottom=2cm,portrait]{geometry}
\usepackage{longtable}
\usepackage{graphicx}
\usepackage{makecell}
\usepackage{multirow}
\usepackage[table]{xcolor}
\usepackage{colortbl}
\usepackage{fancyhdr}
\usepackage[T1]{fontenc}
\usepackage{textcomp}
\usepackage{fontenc}
\usepackage{xltxtra}
\usepackage{tikz}
\usepackage{array}
\usepackage[absolute,overlay]{textpos}

\newcommand{\kn}[1]{
{\fontspec{Kedage}
#1
}}

\usepackage{arydshln}

\arrayrulecolor{gray}

\newcolumntype{P}[1]{>{\centering\arraybackslash}p{#1}}

\fancypagestyle{plain}{
\fancyhf{} % clear all header and footer fields
\lfoot{
\begin{tikzpicture}[overlay]
\node[anchor=south west,yshift=-1cm,xshift=-1cm]
{\includegraphics[height=2cm]{"{{info.imagesdir}}footer_left.png"}};
\end{tikzpicture}}

\rfoot{
\begin{tikzpicture}[overlay]
\node[anchor=south east,yshift=-1cm,xshift=1cm]
{\includegraphics[height=2cm]{"{{info.imagesdir}}footer_right.png"}};
\end{tikzpicture}}

\cfoot{
\fontsize{9}{12}
\begin{tikzpicture}[overlay]
\node[anchor=south west,yshift=0cm,xshift=-6cm]
{\kn{ಗ್ರಾಮ ಪಂಚಾಯ್ತಿ ಮಟ್ಟದ ಗಣಿತ ಸ್ಪರ್ಧೆಯ ವರದಿ} {{info.year}} };
\node[anchor=south west,yshift=-0.5cm,xshift=-6cm]
{\kn{ಹೆಚ್ಚಿನ ಮಾಹಿತಿಗಾಗಿ} - 9845079590 \kn{ಗೆ ಕರೆ ಮಾಡಿ} };
\node[anchor=south,yshift=-0.5cm,xshift=6cm]
{\includegraphics[height=1cm]{"{{info.imagesdir}}footer_logo.png"}};
\end{tikzpicture}}
}


\renewcommand{\headrulewidth}{0pt} % to remove line on header
\renewcommand{\footrulewidth}{0pt} % to remove line on footer

\begin{document}

\thispagestyle{plain}
{
\setlength{\parindent}{0in}
\begin{flushleft}
\kn{ಇವರಿಗೆ},\\~\\
\kn{ಮಾನ್ಯ ಅಧ್ಯಕ್ಷರು,}\\
\kn{ಅಧರಹಳ್ಳಿ ಗ್ರಾಮ ಪಂಚಾಯತಿ}\\
\kn{ಶಿರಹಟ್ಟಿ ತಾಲೂಕು}\\
\kn{ಗದಗ ಜಿಲ್ಲೆ.}\\~\\
\kn{ಮಾನ್ಯರೆ,}\\~\\
\end{flushleft}
\centering{\kn{ವಿಷಯ: ತಮ್ಮ ಗ್ರಾಮ ಪಂಚಾಯತಿ ವ್ಯಾಪ್ತಿಯ ಎಲ್ಲಾ ಸರಕಾರಿ ಪ್ರಾಥಮಿಕ ಶಾಲಾ ಮಕ್ಕಳ ಪೋಷಕರ ಅಭಿಪ್ರಾಯ}\\\kn{ಸಮೀಕ್ಷೆಯ ವರದಿ}}\\
\begin{flushleft}
\kn{ಅಧರಹಳ್ಳಿ ಗ್ರಾಮ ಪಂಚಾಯತಿಯ ಸರಕಾರಿ ಶಾಲೆಗಳಲ್ಲಿ 3 ರಿಂದ 7ನೇ ತರಗತಿಯಲ್ಲಿ ಓದುತ್ತಿರುವ ಮಕ್ಕಳ ಪೋಷಕರ ಅಭಿಪ್ರಾಯ ಸಮೀಕ್ಷೇಯಲ್ಲಿ ಗ್ರಾಮದ ಸ್ವಯಂಸೇವಕರ ಸಹಕಾರದಿಂದ 481 ಪೋಷಕರು ಸಮೀಕ್ಷೆಗೆ ಉತ್ತರಿಸಿರುತ್ತಾರೆ.}\\~\\
\kn{ಪೋಷಕರ ಅಭಿಪ್ರಾಯ ಸಮೀಕ್ಷೆಯಲ್ಲಿ ಮಕ್ಕಳ ಗುಣಾತ್ಮಕ ಕಲಿಕೆ, ಶಾಲಾ ಮೂಲಭೂತ ಸೌಕರ್ಯಗಳು, ಪೋಷಕರು, ಎಸ್.ಡಿ.ಎಮ್.ಸಿ ಮತ್ತು ಶಿಕ್ಷಕರಿಗೆ ಸಂಬಂಧಿಸಿದ 10 ವಿಷಯಗಳ ಬಗ್ಗೆ ಪೋಷಕರ ಅಭಿಪ್ರಾಯ ಕೇಳಲಾಯಿತು.}\\~\\
\kn{ಶಾಲಾವಾರು ಪೋಷಕರ ಅಭಿಪ್ರಾಯ ಸಮೀಕ್ಷೆಯ ವರದಿಯನ್ನು ಈ ಪತ್ರದೊಂದಿಗೆ ಲಗತ್ತಿಸಲಾಗಿದೆ. ಎಸ್.ಡಿ.ಎಮ್.ಸಿ  ಅಧ್ಯಕ್ಷರು ಅಥವಾ ಗ್ರಾಮ ಪಂಚಾಯತಿ ಸದಸ್ಯರ ಜೊತೆಗೆ ವರದಿಯನ್ನು ಹಂಚಿಕೊಳ್ಳಬೇಕೆಂದು ಮತ್ತು ಶಾಲಾ ಮಟ್ಟದಲ್ಲಿ ಚರ್ಚಿಸಿ ಸೂಕ್ತ ಯೋಜನೆಯನ್ನು ರೂಪಿಸಲು ತಿಳಿಸಬೇಕೆಂದು ಕೋರುತ್ತೇವೆ.}\\~\\
\kn{ಈ ಸಮೀಕ್ಷೆಯ ವರದಿಯನ್ನು ಗ್ರಾಮ ಪಂಚಾಯತಿ ಮಟ್ಟದ ಗಣಿತ ಸ್ಪರ್ದೆಯ ವರದಿಯ ಜೊತೆ ಹೋಲಿಕೆ ಮಾಡಿ ನೋಡಲು ಕೋರುತ್ತೇವೆ.}\\~\\
\kn{ಮಕ್ಕಳ ಗುಣಾತ್ಮಕ ಕಲಿಕೆಗೆ ಸಮುದಾಯದ ಭಾಗವಹಿಸುವಿಕಯಲ್ಲಿ, ಎಲ್ಲಾ ಭಾಗೀದಾರರು ಸಮ ಪಾಲುದಾರರು}\\~\\~\\
\kn{ವಂದನೆಗಳೊಂದಿಗೆ}\\
\kn{ತಮ್ಮ ವಿಶ್ವಾಸಿ}\\~\\~\\
\kn{ಜೆ.ವಿ. ಶಂಕರನಾರಾಯಣ}\\
\kn{ಮುಖ್ಯಸ್ಥರು, ಕಾರ್ಯಚರಣೆ ಮತ್ತು ಸಮುದಾಯ ಅಭಿವೃಧ್ಧಿ}\\
\kn{ಅಕ್ಷರ ಫೌಂಡೇಶನ್, ಬೆಂಗಳೂರು}
\end{flushleft}
}
\pagebreak

\thispagestyle{plain}
{
\setlength{\parindent}{0in}
\centering{\large{\kn{ಫೋಷಕರ ಅಭಿಪ್ರಾಯಗಳ ಸಂಗ್ರಹಣಾ ವರದಿ}}}\\~\\~\\
\kn{ಜಿಲ್ಲೆ} : \kn{ {{schoolinfo.district_langname}} }{{schoolinfo.district}} \hfill  \kn{ತಾಲೂಕ} : \kn{ {{schoolinfo.block_langname}} }{{schoolinfo.block}} \\
\kn{ಹಳ್ಳಿ} : {{schoolinfo.village}} \hfill \kn{ತಿಂಗಳು} : {{schoolinfo.month}} \\
\kn{ಶಾಲೆಯ ಹೆಸರು} : {{schoolinfo.schoolname}} \hfill \kn{ಡೈಸ್ ಕೋಡ್} : {{schoolinfo.disecode}}\\~\\
\begin{flushleft}
\kn{ಫೋಷಕರ ಅಭಿಪ್ರಾಯಗಳ ಒಟ್ಟು ಸಂಖ್ಯೆ} : {{schoolinfo.numassessments}}\\~\\~\\
\end{flushleft}
\centering{
\begin{longtable}{|c|P{10cm}|c|c|c|} \hline
\kn{ಕ್ರ.ಸಂ.} & \kn{ಕೇಳಲಾದ ಪ್ರಶ್ನೆಗಳು} & \kn{ಹೌದು} & \kn{ಇಲ್ಲ} & \kn {ಗೊತ್ತಿಲ್ಲ} \\[2ex] \cline{3-5}
&&\multicolumn{3}{|c|}{\kn{ಶೇಕಡವಾರು}}\\ \hline 

{{loop.index}} & \kn{ {{ question.lang_text }} } & {{ question.percentage_yes }} & {{question.percentage_no}} & {{question.percentage_unknown}} \\[4ex] \hline

\end{longtable}

~\\~\\

\begin{tabular}{cc|c}
&\includegraphics[height=1cm]{"{{info.imagesdir}}household_parent.png"}\kn{ಪೋಷಕರ ಗ್ರಹಿಕೆ} & \includegraphics[height=1cm]{"{{info.imagesdir}}household_clipboard.png"} \kn{ವಾಸ್ತವಿಕತೆ (ಗ್ರಾ, ಪಂ ಗಣಿತ ಸ್ಪರ್ಧೆ)}\\ \cline{2-3}
\includegraphics[height=1cm]{"{{info.imagesdir}}household_addition.png"} \kn{(ಸಂಕಲನ)}& {{compare.parent.Addition}} &  {{compare.gpcontest.Addition}}\\ 
\includegraphics[height=1cm]{"{{info.imagesdir}}household_subtraction.png"} \kn{(ವ್ಯವಕಲನ)} & {{compare.parent.Subtraction}} & {{compare.gpcontest.Subtraction}}\\ \cline{2-3}
\end{tabular}
}
~\\~\\

}
\end{document}