\documentclass[12pt]{article}
\usepackage[a4paper,top=2cm,left=1cm,right=1cm,bottom=2cm,portrait]{geometry}
\usepackage{longtable}
\usepackage{graphicx}
\usepackage{makecell}
\usepackage{multirow}
\usepackage[table]{xcolor}
\usepackage{colortbl}
\usepackage{fancyhdr}
\usepackage[T1]{fontenc}
\usepackage{textcomp}
\usepackage{fontenc}
\usepackage{xltxtra}
\usepackage{tikz}
\usepackage{array}
\usepackage[absolute,overlay]{textpos}

\newcommand{\kn}[1]{
{\fontspec{Kedage}
#1
}}

\newcolumntype{P}[1]{>{\centering\arraybackslash}p{#1}}

\fancypagestyle{plain}{
\fancyhf{} % clear all header and footer fields
\lfoot{
\begin{tikzpicture}[overlay]
\node[anchor=south west,yshift=-1cm,xshift=-1cm]
{\includegraphics[height=2cm]{"{{info.imagesdir}}footer_left.png"}};
\end{tikzpicture}}

\rfoot{
\begin{tikzpicture}[overlay]
\node[anchor=south east,yshift=-1cm,xshift=1cm]
{\includegraphics[height=2cm]{"{{info.imagesdir}}footer_right.png"}};
\end{tikzpicture}}

\cfoot{
\fontsize{9}{12}
\begin{tikzpicture}[overlay]
\node[anchor=south west,yshift=0cm,xshift=-6cm]
{\kn{ಗ್ರಾಮ ಪಂಚಾಯ್ತಿ ಮಟ್ಟದ ಗಣಿತ ಸ್ಪರ್ಧೆಯ ವರದಿ} {{info.acadyear}} };
\node[anchor=south west,yshift=-0.5cm,xshift=-6cm]
{\kn{ಹೆಚ್ಚಿನ ಮಾಹಿತಿಗಾಗಿ} - 9845079590 \kn{ಗೆ ಕರೆ ಮಾಡಿ} };
\node[anchor=south,yshift=-0.5cm,xshift=6cm]
{\includegraphics[height=1cm]{"{{info.imagesdir}}footer_logo.png"}};
\end{tikzpicture}}
}

%% Headers and footers
\fancypagestyle{title}{
\fancyhf{} % clear all header and footer fields
\chead{
\begin{tikzpicture}[overlay]
\node[anchor=north, yshift=1.5cm, xshift=0cm]
{\includegraphics[width=3.5cm]{"{{info.imagesdir}}title_header_center.png"}};
\end{tikzpicture}}

\lhead{
\begin{tikzpicture}[overlay]
\node[anchor=north west,yshift=1.5cm,xshift=-1cm]
{\includegraphics[width=5cm]{"{{info.imagesdir}}title_header_left.png"}};
\end{tikzpicture}}

\rhead{
\begin{tikzpicture}[overlay]
\node[anchor=north east,yshift=1.5cm,xshift=1cm]
{\includegraphics[width=3cm]{"{{info.imagesdir}}title_header_right.png"}};
\end{tikzpicture}}

\begin{tikzpicture}[overlay]
\node[anchor=north west, yshift=-6.5cm, xshift=5cm]
{\includegraphics[width=7cm]{"{{info.imagesdir}}title_text.png"}};
\end{tikzpicture}

\begin{tikzpicture}[overlay]
%\node[anchor=south west, yshift=200pt,xshift=5cm]
\node[anchor=north west, yshift=-16.5cm,xshift=6cm]
{\includegraphics[width=4cm]{"{{info.imagesdir}}logo.jpg"}};
\end{tikzpicture}

\lfoot{
\begin{tikzpicture}[overlay]
\node[anchor=south west,yshift=0cm,xshift=-1cm]
{\includegraphics[height=4cm]{"{{info.imagesdir}}title_footer_left.png"}};
\end{tikzpicture}}

\rfoot{
\begin{tikzpicture}[overlay]
\node[anchor=south east,yshift=0cm,xshift=1cm]
{\includegraphics[height=3cm]{"{{info.imagesdir}}title_footer_right.png"}};
\end{tikzpicture}}

\setlength{\headheight}{15pt}
%%\setlength{\footskip}{120pt}
}


\renewcommand{\headrulewidth}{0pt} % to remove line on header
\renewcommand{\footrulewidth}{0pt} % to remove line on footer




\begin{document}

\begin{titlepage}
    \thispagestyle{title}
    \begin{center}
    \vspace*{10.5cm}
     \textbf{\huge \kn{ಗ್ರಾಮ ಪಂಚಾಯ್ತಿ ಮಟ್ಟದ}} \\~\\
        \textbf{\huge\kn{ಗಣಿತ ಸ್ಪರ್ಧೆಯ ವರದಿ}} \\~\\
	    \Large{{info.acadyear}}
    \end{center}
\end{titlepage}
\pagebreak

\thispagestyle{plain}
{
\setlength{\parindent}{0in}
~\\~\\~\\
\kn{ಇವರಿಗೆ,} \hfill  \kn{ {{info.month}} } {{info.year}} \\ [2ex]
\kn{ಮಾನ್ಯ ಅಧ್ಯಕ್ಷರು,}\\ [1ex]
\kn{ {{gpinfo.gp_langname}} } {{gpinfo.gpname}} \kn{ ಗ್ರಾಮ ಪಂಚಾಯ್ತಿ,}\\ [1ex]
\kn{ {{gpinfo.block_langname}} } {{gpinfo.block}} \kn{ ತಾಲೂಕು,}\\ [1ex]
\kn{ {{gpinfo.district_langname}} } {{gpinfo.district}} \kn{ ಜಿಲ್ಲೆ.
\\~\\ [2ex]
ಮಾನ್ಯರೆ,
\\~\\
ವಿಷಯ: ಗಣಿತ ಕಲಿಕಾ ಆಂದೋಲನದ ಪರವಾಗಿ} \kn{ {{gpinfo.gp_langname}} } {{gpinfo.gpname}} \kn{ ಗ್ರಾಮ ಪಂಚಾಯ್ತಿ ಮತ್ತು ಸಮುದಾಯದವರಿಗೆ ಅಭಿನಂದನೆಗಳು.
\\~\\
ಗ್ರಾಮೀಣಾಭಿವೃಧ್ಧಿ ಮತ್ತು ಪಂಚಾಯತ್‌ ರಾಜ್‌ ಇಲಾಖೆ,ಸಮುದಾಯ, ಶಾಲಾ ಶಿಕ್ಷಣ ಸಾಕ್ಷರತಾ ಇಲಾಖೆ ಮತ್ತು ಅಕ್ಷರ ಫೌಂಡೇಶನ್ ಸಹಭಾಗಿತ್ವದಲ್ಲಿ ದಿನಾಂಕ}{{gpinfo.contestdate}} \kn{ರಂದು ಗ್ರಾಮ ಪಂಚಾಯತಿ ಮಟ್ಟದಲ್ಲಿ 4, 5 ಮತ್ತು 6ನೇ ತರಗತಿಯ ಶಾಲಾ ಮಕ್ಕಳ ಗಣಿತ ಸ್ಪರ್ಧೆಯನ್ನು ನಿಮ್ಮ ಅಧ್ಯಕ್ಷತೆಯಲ್ಲಿ ಯಶಸ್ವಿಯಾಗಿ ನಡೆಸಿರುವುದಕ್ಕೆ ಅಭಿನಂದನೆಗಳನ್ನು ಸಲ್ಲಿಸುತ್ತೇವೆ. ಈ ಸ್ಪರ್ಧೆಯನ್ನು ಆಯೋಜಿಸುವ ಮೂಲಕ ಗ್ರಾಮ ಪಂಚಾಯ್ತಿಯು ಸಮುದಾಯದಲ್ಲಿ ಮಕ್ಕಳ ಕಲಿಕಾ ಗುಣಮಟ್ಟ ಹೆಚ್ಚಿಸಲು ಒಂದು ಮಹತ್ವವಾದ ಹೆಜ್ಜೆಯನ್ನು ಇಟ್ಟಿದೆ.}
\\~\\
\kn{ಗಣಿತ ಕಲಿಕಾ ಆಂದೋಲನದ ಈ ಗಣಿತ ಸ್ಪರ್ಧೆಯಲ್ಲಿ ತಮ್ಮ ಗ್ರಾಮ ಪಂಚಾಯ್ತಿ ವ್ಯಾಪ್ತಿಯ ಎಲ್ಲಾ ಗ್ರಾಮಗಳಿಂದ ಒಟ್ಟು}{{gpinfo.totalstudents}} \kn{ಮಕ್ಕಳು ಭಾಗವಹಿಸಿರುತ್ತಾರೆ. ಎಲ್ಲಾ ಗ್ರಾಮಗಳ ಗ್ರಾಮ ಪಂಚಾಯ್ತಿ ಸದಸ್ಯರು, ಪೋಷಕರು,} SDMC \kn{ಅಧ್ಯಕ್ಷರು ಮತ್ತು ಸದಸ್ಯರು, ಸ್ವಯಂಸೇವಕರು, ಸಮುದಾಯದ ಮುಖಂಡರು, ಶಿಕ್ಷಕರು ಹಾಗೂ ಸ್ಥಳೀಯ ಸಂಘ}-\kn{ಸಂಸ್ಥೆಗಳು ಶಾಲಾ ಮಕ್ಕಳ ಗಣಿತ ಸ್ಪರ್ಧೆಗೆ ಸಂಪನ್ಮೂಲ, ಸಹಕಾರ ಹಾಗೂ ಸಹಭಾಗಿತ್ವದ ರೂಪದಲ್ಲಿ ಭಾಗವಹಿಸಿ ಯಶಸ್ವಿಗೊಳಿಸಿರುವುದಕ್ಕೆ ಎಲ್ಲಾ ಭಾಗೀದಾರರಿಗೂ ಧನ್ಯವಾದಗಳು.
\\~\\
ಗ್ರಾಮ ಪಂಚಾಯ್ತಿ ಅಧ್ಯಕ್ಷರ ನಾಯಕತ್ವದಲ್ಲಿ ಸ್ಥಳೀಯ ಶಾಲಾ ಮಕ್ಕಳ ಗುಣಾತ್ಮಕ ಶಿಕ್ಷಣಕ್ಕಾಗಿ ಇಂತಹ ಶೈಕ್ಷಣಿಕ ಕಾರ್ಯಕ್ರಮಗಳನ್ನು ನಿಯಮಿತವಾಗಿ ಮಾಡಬೇಕೆಂದು ಕೋರುತ್ತೇವೆ.
\\~\\
ಈ ಪತ್ರದೊಂದಿಗೆ ನಿಮ್ಮ ಗ್ರಾಮ ಪಂಚಾಯ್ತಿ ಮಟ್ಟದಲ್ಲಿ ನಡೆದ ಶಾಲಾ ಮಕ್ಕಳ ಗಣಿತ ಸ್ಪರ್ಧೆಯ ಸಂಕ್ಷಿಪ್ತ ವರದಿಯನ್ನು ಮುಂದಿನ ಯೋಜನೆ ಮತ್ತು ಸೂಕ್ತ ಕ್ರಮಕ್ಕಾಗಿ ಲಗತ್ತಿಸಲಾಗಿದೆ.
\\~\\~\\
ವಂದನೆಗಳೊಂದಿಗೆ
\\~\\~\\~\\
\begin{tikzpicture}[overlay]
\node[anchor=south east,yshift=0.5cm,xshift=3cm]
{\includegraphics[height=1.5cm]{"{{info.imagesdir}}Ashoks_signature.png"}};
\end{tikzpicture}
ಅಶೋಕ್ ಕಾಮತ್,
\\
ಅಧ್ಯಕ್ಷರು ,ಅಕ್ಷರ ಫೌಂಡೇಶನ್\\
ಬೆಂಗಳೂರು\\
}
}

\pagebreak



\thispagestyle{plain}
{
\setlength{\parindent}{0in}

\kn{ಗ್ರಾಮ ಪಂಚಾಯ್ತಿ} : \kn{ {{gpinfo.gp_langname}} } {{gpinfo.gpname}} \hfill \kn{ಜಿಲ್ಲೆ} : \kn{ {{gpinfo.district_langname}} } {{gpinfo.district}}\ \\ [1ex] 
\kn{ತಾಲೂಕ} : \kn{ {{gpinfo.block_langname}} } {{gpinfo.block}}\hfill \kn{ಒಟ್ಟು ಶಾಲೆಗಳ ಸಂಖ್ಯೆ} : {{gpinfo.school_count}} \\~\\~\\~\\

~\\~\\

\kn{ಗ್ರಾಮ ಪಂಚಾಯ್ತಿ , ಶಿಕ್ಷಣ ಇಲಾಖೆ, ಸಮುದಾಯ ಮತ್ತು ಅಕ್ಷರ ಫೌಂಡೇಶನ್ ಸಹಭಾಗಿತ್ವದಲ್ಲಿ ದಿನಾಂಕ} {{gpinfo.contestdate}} \kn{ರಂದು ಗ್ರಾಮ ಪಂಚಾಯ್ತಿ ಮಟ್ಟದಲ್ಲಿ ನಡೆದ 4, 5 ಮತ್ತು 6ನೇ ತರಗತಿಯ ಶಾಲಾ ಮಕ್ಕಳ ಗಣಿತ ಸ್ಪರ್ಧೆಯ ಕಿರು ವರದಿ ಈ ಕೆಳಕಂಡಂತಿದೆ. \\ [1ex]
ಈ ಗ್ರಾಮ ಪಂಚಾಯ್ತಿ ವ್ಯಾಪ್ತಿಯಲ್ಲಿ ಸರ್ಕಾರಿ ಶಾಲೆಗಳಿಂದ 4, 5 ಮತ್ತು 6ನೇ ತರಗತಿಯ ಒಟ್ಟು}{{gpinfo.totalstudents}}\kn{ಮಕ್ಕಳು ಗ್ರಾಮ ಪಂಚಾಯ್ತಿ ಮಟ್ಟದ ಗಣಿತ ಸ್ಪರ್ಧೆಯಲ್ಲಿ ಭಾಗವಹಿಸಿದ್ದರು.}
\\~\\
\kn{ತರಗತಿವಾರು ಗಣಿತ ಸ್ಪರ್ಧೆಯಲ್ಲಿ ಭಾಗವಹಿಸಿದ್ದ ಮಕ್ಕಳ ಒಟ್ಟಾರೆ ಫಲಿತಾಂಶಗಳು ಈ ಕೆಳಗಿನಂತೆ ಇವೆ.}\\


\begin{longtable}{|P{3cm}|P{4cm}|P{2.2cm}|P{2.2cm}|P{2.2cm}|P{2.2cm}|} \hline
\textbf{\kn{ತರಗತಿ}} & \textbf{\kn{ಒಟ್ಟು ಮಕ್ಕಳ ಸಂಖ್ಯೆ}} & \multicolumn{4}{|c|}{\textbf{\kn{ಅಂಕ ಗಳಿಸಿದ ಮಕ್ಕಳ ಸಂಖ್ಯೆ (ಶೇಕಡವಾರು)}}}\\ \cline{3-6}
\rule{0cm}{0.3cm}& &  \textbf{<35\%} & \textbf{35\%-59\%}  & \textbf{60\%-74\%} & \textbf{75\%-100\%} \\ \hline \endhead

\rule{0cm}{0.3cm} {{ assessment["class"] }}\kn{ನೇ ತರಗತಿ} & {{ assessment["overall_scores"]["total"]}} & {{assessment["overall_scores"]["below35"]}} & {{assessment["overall_scores"]["35to59"]}} & {{assessment["overall_scores"]["60to74"]}} & {{assessment["overall_scores"]["75to100"]}} \\ \hline

\end{longtable}

\begingroup
\fontsize{11}{12}
{
\setlength\LTleft{-0.1cm}
\setlength\LTright{-0.1cm}
\begin{longtable}{|P{2.5cm}|P{3.25cm}|c|c|c|c|c|c|} \hline
\textbf{\kn{ತರಗತಿ}} & \textbf{\kn{ಒಟ್ಟು ಮಕ್ಕಳ ಸಂಖ್ಯೆ}} & \multicolumn{6}{|c|}{\textbf{\kn{ಮೂಲ ಕ್ರಿಯೆಗಳನ್ನು ಸರಿಯಾಗಿ ಮಾಡಬಲ್ಲ ಮಕ್ಕಳು}}} \\ \cline{3-8}
\rule{0cm}{0.3cm} &&\textbf{\kn{ಸಂಖ್ಯಾ ಪರಿಚಯ}}&\textbf{\kn{ಸ್ಥಾನ ಬೆಲೆ}}&\textbf{\kn{ಸಂಕಲನ}}&\textbf{\kn{ವ್ಯವಕಲನ}}&\textbf{\kn{ಗುಣಾಕಾರ}}&\textbf{\kn{ಭಾಗಾಕಾರ}} \\ \hline \endhead

\rule{0cm}{0.3cm} {{assessment["class"]}}\kn{ನೇ ತರಗತಿ} & {{assessment["competency_scores"]["total"]}} & {{assessment["competency_scores"]["Number Recognition"]}}  & {{assessment["competency_scores"]["Place Value"]}} & {{assessment["competency_scores"]["Addition"]}} & {{assessment["competency_scores"]["Subtraction"]}} & {{assessment["competency_scores"]["Multiplication"]}} & {{assessment["competency_scores"]["Division"]}} \\ \hline

\end{longtable}
}
\endgroup
 

{
~\\
\centering{\kn{5 ವರ್ಷ ಶಾಲಾ ದಿನಗಳನ್ನು ಮುಗಿಸಿದ (6ನೇ ತರಗತಿ) ಮಕ್ಕಳ ಒಟ್ಟಾರೆ ಫಲಿತಾಂಶ ಶೇಕಡವಾರು ಈ ಕೆಳಗಿನಂತಿದೆ.}} \\[2ex]
 
\begingroup
\fontsize{8.75}{12}
{

{
\centering{
\begin{longtable}{|c|c|c|c|c|c|} \hline
\cellcolor{gray}{\textbf{\makecell[b]{\kn{ಸಂಖ್ಯಾ ಪರಿಚಯವನ್ನು} \\ \kn{ಸರಿಯಾಗಿ ಮಾಡಬಲ್ಲ ಮಕ್ಕಳು}}}} & \cellcolor{lightgray}{\textbf{\makecell[b]{\kn{ಸ್ಥಾನ ಬೆಲೆಯನ್ನು ಸರಿಯಾಗಿ}\\ \kn{ಮಾಡಬಲ್ಲ ಮಕ್ಕಳು}}}} & \textbf{\makecell[b]{\kn{ಸಂಕಲನವನ್ನು ಸರಿಯಾಗಿ}\\ \kn{ಮಾಡಬಲ್ಲ ಮಕ್ಕಳು}}} & \cellcolor{gray}{\textbf{\makecell[b]{\kn{ವ್ಯವಕಲನವನ್ನು ಸರಿಯಾಗಿ}\\ \kn{ಮಾಡಬಲ್ಲ ಮಕ್ಕಳು}}}} & \cellcolor{lightgray}{\textbf{\makecell[b]{\kn{ಗುಣಾಕಾರವನ್ನು ಸರಿಯಾಗಿ}\\ \kn{ಮಾಡಬಲ್ಲ ಮಕ್ಕಳು}}}} & \textbf{\makecell[b]{\kn{ಭಾಗಾಕಾರವನ್ನು ಸರಿಯಾಗಿ}\\ \kn{ಮಾಡಬಲ್ಲ ಮಕ್ಕಳು}}} \\ \hline \endhead

\rule{0cm}{0.3cm} {{percent_scores["assessments"][assessment]["Number Recognition"]}}&  {{percent_scores["assessments"][assessment]["Place Value"]}} & {{percent_scores["assessments"][assessment]["Addition"]}} & {{percent_scores["assessments"][assessment]["Subtraction"]}} & {{percent_scores["assessments"][assessment]["Multiplication"]}} &  {{percent_scores["assessments"][assessment]["Division"]}}  \\ \hline

\end{longtable}
}
}

{
\centering{
\begin{longtable}{|c|c|c|c|} \hline
 \textbf{\makecell[b]{\kn{ಸಂಕಲನವನ್ನು ಸರಿಯಾಗಿ}\\ \kn{ಮಾಡಬಲ್ಲ ಮಕ್ಕಳು}}} & \cellcolor{gray}{\textbf{\makecell[b]{\kn{ವ್ಯವಕಲನವನ್ನು ಸರಿಯಾಗಿ}\\ \kn{ಮಾಡಬಲ್ಲ ಮಕ್ಕಳು}}}} & \cellcolor{lightgray}{\textbf{\makecell[b]{\kn{ಗುಣಾಕಾರವನ್ನು ಸರಿಯಾಗಿ}\\ \kn{ಮಾಡಬಲ್ಲ ಮಕ್ಕಳು}}}} & \textbf{\makecell[b]{\kn{ಭಾಗಾಕಾರವನ್ನು ಸರಿಯಾಗಿ}\\ \kn{ಮಾಡಬಲ್ಲ ಮಕ್ಕಳು}}} \\ \hline \endhead

\rule{0cm}{0.3cm} {{percent_scores["assessments"][assessment]["Addition"]}} & {{percent_scores["assessments"][assessment]["Subtraction"]}} & {{percent_scores["assessments"][assessment]["Multiplication"]}} &  {{percent_scores["assessments"][assessment]["Division"]}}  \\ \hline
	
\end{longtable}
}
}


}
\endgroup
}


\begin{flushleft}
\kn{(ಸೂಚನೆ : ಈ ವರದಿಯ ಕೋಷ್ಟಕಗಳಲ್ಲಿ ದಶಕ ಸಹಿತ ಸಂಕಲನ, ದಶಕ ಸಹಿತ ವ್ಯವಕಲನ ಹಾಗು ಶೇಷ ಸಹಿತ ಭಾಗಾಕಾರವನ್ನು ವಿಶ್ಲೇಷಣೆಗೆ ಪರಿಗಣಿಸಲಾಗಿದೆ)}
~\\~\\
\kn{ಈ ಮೇಲ್ಕಂಡ ಫಲಿತಾಂಶಗಳು ಇನ್ನೂ ಉತ್ತಮ ಪಡಿಸಲು ಗ್ರಾಮ ಪಂಚಾಯ್ತಿ ಜೊತೆಗೆ ಇತರೆ ಭಾಗೀದಾರರಾದ ನಾಗರಿಕ ಸೌಕರ್ಯ ಸಮಿತಿ} (CIVIL Amenities Committee), SDMC, \kn{ಪೋಷಕರು,ಶೈಕ್ಷಣಿಕ ಸ್ವಯಂಸೇವಕರು ಮತ್ತು ಶಾಲಾ ಶಿಕ್ಷಕರ ಜೊತೆಗೂಡಿ ಯೋಜನೆಯನ್ನು ರೂಪಿಸುತ್ತೀರಾ ಎಂದು ಭಾವಿಸುತ್ತೇವೆ.} \\[5ex]
\end{flushleft}


\centering{\kn{ವಂದನೆಗಳೊಂದಿಗೆ}} \\[10ex]

~\\~\\


\begin{flushleft}
\kn{ಇಂದ,}\\
\kn{ಅಕ್ಷರ ಫೌಂಡೇಶನ್}
\end{flushleft}
}
\end{document}
