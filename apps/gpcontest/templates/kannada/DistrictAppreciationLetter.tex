\documentclass[12pt]{article}
\usepackage[a4paper,top=2cm,left=1cm,right=1cm,bottom=2cm,portrait]{geometry}
\usepackage{longtable}
\usepackage{graphicx}
\usepackage{makecell}
\usepackage{multirow}
\usepackage[table]{xcolor}
\usepackage{colortbl}
\usepackage{fancyhdr}
\usepackage[T1]{fontenc}
\usepackage{textcomp}
\usepackage{fontenc}
\usepackage{xltxtra}
\usepackage{tikz}
\usepackage{array}
\usepackage[absolute,overlay]{textpos}

\newcommand{\kn}[1]{
{\fontspec{Kedage}
#1
}}

\newcolumntype{P}[1]{>{\centering\arraybackslash}p{#1}}

\fancypagestyle{plain}{
\fancyhf{} % clear all header and footer fields

\chead{
\begin{tikzpicture}[overlay]
\node[anchor=north, yshift=1.3cm, xshift=0cm]
{\includegraphics[height=3cm]{"{{info.imagesdir}}title_text.png"}};
\end{tikzpicture}}

\lhead{
\begin{tikzpicture}[overlay]
\node[anchor=north west,yshift=1.15cm,xshift=-1cm]
{\includegraphics[width=2cm]{"{{info.imagesdir}}appreciation_header_left.png"}};
\end{tikzpicture}}

\rhead{
\begin{tikzpicture}[overlay]
\node[anchor=north east,yshift=1.15cm,xshift=1cm]
{\includegraphics[width=2cm]{"{{info.imagesdir}}appreciation_header_right.png"}};
\end{tikzpicture}}


\lfoot{
\begin{tikzpicture}[overlay]
\node[anchor=south west,yshift=-1cm,xshift=-1cm]
{\includegraphics[width=3cm]{"{{info.imagesdir}}appreciation_footer_left.png"}};
\end{tikzpicture}}

\cfoot{
\begin{tikzpicture}[overlay]
\node[anchor=north, yshift=1.5cm, xshift=0cm]
{\includegraphics[height=2cm]{"{{info.imagesdir}}appreciation_header_center.png"}};
\end{tikzpicture}}


\rfoot{
\begin{tikzpicture}[overlay]
\node[anchor=south east,yshift=-1cm,xshift=1cm]
{\includegraphics[width=2cm]{"{{info.imagesdir}}appreciation_footer_right.png"}};
\end{tikzpicture}}

}



\renewcommand{\headrulewidth}{0pt} % to remove line on header
\renewcommand{\footrulewidth}{0pt} % to remove line on footer
\begin{document}


\thispagestyle{plain}
{
\setlength{\parindent}{0in}
~\\~\\~\\~\\
\centering{\textbf{\huge{\kn{ಅಭಿವಂದನಾ ಪತ್ರ}} \\~\\ 
\LARGE{\kn{ಗ್ರಾಮ ಪಂಚಾಯತಿ ಮಟ್ಟದ ಗಣಿತ ಸ್ಪರ್ಧೆ}}}}\\~\\~\\ 
\begin{flushleft}
\kn{ಗ್ರಾಮೀಣ ಅಭಿವೃಧ್ಧಿ ಮತ್ತು ಪಂಚಾಯತ್ ರಾಜ್ ಇಲಾಖೆ, ಸಾರ್ವಜನಿಕ ಶಿಕ್ಷಣ ಇಲಾಖೆ, ಸಮುದಾಯ, ಶೈಕ್ಷಣಿಕ ಸ್ವಯಂಸೇವಕರು, ಗಣಿತ ಸ್ಪರ್ಧಾ ನಾಯಕರು ಹಾಗೂ ಅಕ್ಷರ ಫೌಂಡೇಶನ್ ಸಹಭಾಗಿತ್ವದಲ್ಲಿ ಜುಲೈ} 2019 \kn{ರಿಂದ ಫೆಬ್ರವರಿ} 2020 \kn{ರ ಅವಧಿಯಲ್ಲಿ ಕರ್ನಾಟಕದ} 12 \kn{ಜಿಲ್ಲೆಗಳಿಂದ} {{info.state.num_gps}} \kn{ಗ್ರಾಮ ಪಂಚಾಯತಿಗಳಲ್ಲಿ} {{info.state.num_schools}}  \kn{ಸರ್ಕಾರಿ ಪ್ರಾಥಮಿಕ ಶಾಲೆಗಳ} 4,5 \kn{ಮತ್ತು} 6 \kn{ನೇ ತರಗತಿಯಲ್ಲಿ ಓದುತ್ತಿರುವ} {{info.state.num_students}} \kn{ಮಕ್ಕಳು ಗ್ರಾಮ ಪಂಚಾಯತಿ ಮಟ್ಟದ ಗಣಿತ ಸ್ಪರ್ಧೆಯಲ್ಲಿ ಭಾಗವಹಿಸಿರುತ್ತಾರೆ. ಇದೊಂದು ಅತೀ ವಿಶೇಷ ಹಾಗೂ ಯಶಸ್ವಿ ಗಣಿತ ಕಲಿಕಾ ಆಂದೋಲನವಾಗಿರುತ್ತದೆ.}
\\~\\
\kn{ {{info.district.boundary_langname}} }\kn{ಜಿಲ್ಲೆಯಲ್ಲಿ} {{info.district.num_blocks}} \kn{ತಾಲೂಕಿನಿಂದ} {{info.district.num_gps}} \kn{ಗ್ರಾಮ ಪಂಚಾಯತಿಯಿಂದ} {{info.district.num_schools}} \kn{ಸರ್ಕಾರಿ ಪ್ರಾಥಮಿಕ ಶಾಲೆಗಳ} 4,5 \kn{ಮತ್ತು} 6\kn{ನೇ ತರಗತಿಯಲ್ಲಿ ಓದುತ್ತಿರುವ} {{info.district.num_students}} \kn{ಮಕ್ಕಳು ಗಣಿತ ಸ್ಪರ್ಧೆಯಲ್ಲಿ ಭಾಗವಹಿಸಿರುತ್ತಾರೆ ಎಂದು ತಿಳಿಸಲು ಸಂತೋಷವಾಗುತ್ತದೆ.}
\\~\\
\kn{ಶ್ರೀ/ಶ್ರೀಮತಿ}\textbf{\kn{ {{info.designation_name}} }},\kn{ {{info.designation}} },\kn{ {{info.district.boundary_langname}} }\kn{ಜಿಲ್ಲೆಯ  ಸರ್ಕಾರಿ ಪ್ರಾಥಮಿಕ ಶಾಲಾ ಮಕ್ಕಳ ಗುಣಾತ್ಮಕ ಶಿಕ್ಷಣಕ್ಕಾಗಿ ನಿರ್ವಹಿಸುತ್ತಿರುವ ಪಾತ್ರ, ನೀಡುತ್ತಿರುವ ಪ್ರೇರಣೆ ಹಾಗೂ ಸಹಕಾರವನ್ನು ಗುರುತಿಸಿ ಈ ಸಂದರ್ಭದಲ್ಲಿ ಅಕ್ಷರ ಫೌಂಡೇಶನ್, ಬೆಂಗಳೂರು ಹೃದಯಪೂರ್ವಕ ಅಭಿವಂದನೆಗಳನ್ನು ಸಲ್ಲಿಸುತ್ತದೆ.}
\\~\\
\kn{ಪ್ರತೀ ವರ್ಷ ಎಲ್ಲಾ ಗ್ರಾಮ ಪಂಚಾಯತಿಗಳಲ್ಲಿ ಗಣಿತ ಸ್ಪರ್ಧೆಯನ್ನು ಮುಂದುವರೆಸಿಕೊಂಡು ಹೋಗಲು ಬೇಕಾಗುವ ಸಹಭಾಗಿತ್ವದ ಪರಿಸರವನ್ನು ರೂಪಿಸಬೇಕೆಂದು ವಿನಂತಿಸಿಕೊಳ್ಳುತ್ತೇವೆ.}
\\~\\~\\~\\
\end{flushleft}

\centering{\kn{ಆತ್ಮೀಯ ವಂದನೆಗಳೊಂದಿಗೆ}\\~\\~\\~\\~\\
{\includegraphics[height=1cm]{"{{info.imagesdir}}a_signature.png"}} \hfill {\includegraphics[height=1cm]{"{{info.imagesdir}}s_signature.png"}}\rule{3.8cm}{0cm}\\
\kn{ಅಶೋಕ್ ಕಾಮತ್}\hfill \kn{ಜೆ.ವಿ. ಶಂಕರನಾರಾಯಣ} \rule{4.7cm}{0cm}\\
\kn{ಅಧ್ಯಕ್ಷರು, ಅಕ್ಷರ ಫೌಂಡೇಶನ್}\hfill \kn{ಮುಖ್ಯಸ್ಥರು, ಕಾರ್ಯಾಚರಣೆ ಮತ್ತು ಸಮುದಾಯ ಅಭಿವೃದ್ಧಿ}\\
\kn{ಬೆಂಗಳೂರು}\hfill \kn{ಅಕ್ಷರ ಫೌಂಡೇಶನ್, ಬೆಂಗಳೂರು}\rule{3.6cm}{0cm}

\end{document}
