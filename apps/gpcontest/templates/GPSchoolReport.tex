\documentclass[10pt]{article}
\usepackage[a4paper,top=1cm,left=2cm,right=2cm,bottom=2cm,portrait]{geometry}
\usepackage{longtable}
\usepackage{graphicx}
\usepackage{makecell}
\usepackage{multirow}
\usepackage{xcolor}
\usepackage{colortbl}
\usepackage{fancyhdr}
\usepackage[T1]{fontenc}
\usepackage{textcomp}
\usepackage{fontenc}
\usepackage{xltxtra}
\usepackage{tikz}
\usepackage{array}

\newcommand{\kn}[1]{
{\fontspec{Kedage}
#1
}}

\newcommand\includegraphicsifexists[2][height=1.5cm]{\IfFileExists{#2}{\includegraphics[#1]{#2}}{}}

\newcolumntype{P}[1]{>{\centering\arraybackslash}p{#1}}

\fancypagestyle{plain}{
\fancyhf{} % clear all header and footer fields
\lfoot{
\begin{tikzpicture}[overlay]
\node[anchor=south west,yshift=-1cm,xshift=-2cm]
{\includegraphics[height=2cm]{"{{info.imagesdir}}footer_left.png"}};
\end{tikzpicture}}

\rfoot{
\begin{tikzpicture}[overlay]
\node[anchor=south east,yshift=-1cm,xshift=2cm]
{\includegraphics[height=2cm]{"{{info.imagesdir}}footer_right.png"}};
\end{tikzpicture}}

\cfoot{
\fontsize{9}{12}
\begin{tikzpicture}[overlay]
\node[anchor=south west,yshift=0cm,xshift=-6cm]
{\kn{ಗ್ರಾಮ ಪಂಚಾಯ್ತಿ ಮಟ್ಟದ ಗಣಿತ ಸ್ಪರ್ಧೆಯ ವರದಿ} {{info.year}} };
\node[anchor=south west,yshift=-0.5cm,xshift=-6cm]
{\kn{ಹೆಚ್ಚಿನ ಮಾಹಿತಿಗಾಗಿ} - 9845079590 \kn{ಗೆ ಕರೆ ಮಾಡಿ} };
\node[anchor=south,yshift=-0.5cm,xshift=6cm]
{\includegraphics[height=1cm]{"{{info.imagesdir}}footer_logo.png"}};
\end{tikzpicture}}
}

\renewcommand{\headrulewidth}{0pt} % to remove line on header
\renewcommand{\footrulewidth}{0pt} % to remove line on footer

\begin{document}

\thispagestyle{plain}
{
\setlength{\parindent}{0in}

\kn{ಜಿಲ್ಲೆ} : \kn{ {{schoolinfo.district_langname}} } {{schoolinfo.district}} \hfill  GP Id : {{schoolinfo.gpid}} \\
\kn{ತಾಲೂಕು} : \kn{ {{schoolinfo.block_langname}} } {{schoolinfo.block}} \hfill DISE Code : {{schoolinfo.disecode}} \\
\kn{ಗ್ರಾಮ ಪಂಚಾಯ್ತಿ} : \kn{ {{schoolinfo.gp_langname}} } {{schoolinfo.gpname}}\hfill \kn{ಶಾಲಾ ಹೆಸರು} : {{schoolinfo.schoolname}}\\

\kn{ಗ್ರಾಮ ಪಂಚಾಯ್ತಿ , ಶಿಕ್ಷಣ ಇಲಾಖೆ, ಸಮುದಾಯ ಮತ್ತು ಅಕ್ಷರ ಫೌಂಡೇಶನ್ ಸಹಭಾಗಿತ್ವದಲ್ಲಿ ದಿನಾಂಕ} {{schoolinfo.contestdate}} \kn{ರಂದು ಗ್ರಾಮ ಪಂಚಾಯ್ತಿ ಮಟ್ಟದಲ್ಲಿ ನಡೆದ ನಿಮ್ಮ ಶಾಲಾ ಮಕ್ಕಳ ಗಣಿತ ಸ್ಪರ್ಧೆಯ ತರಗತಿ ವಾರು ಕಿರು ವರದಿ ಈ ಕೆಳಕಂಡಂತಿದೆ}

\begingroup
\fontsize{9}{12}
{
\setlength\LTleft{-.7cm}
\setlength\LTright{-.7cm}
\renewcommand\arraystretch{1.45}
\begin{longtable}{|P{1cm}|p{10cm}|P{3cm}|P{3cm}|} \hline
\multicolumn{4}{|c|}{ {\large \bf {{info.classname}} } \kn{ನೇ ತರಗತಿ}} \\ \hline
\multicolumn{4}{|c|}{\kn{ಮಕ್ಕಳ ಒಟ್ಟು ಸಂಖ್ಯೆ} : {{assessmentinfo.num_students}}} \\ \hline
\textbf{\kn{ಕ್ರಮ ಸಂಖ್ಯೆ}} & \textbf{\kn{ಪ್ರಶ್ನೆಗಳು}} & \textbf{\kn{ಸರಿಯಾಗಿ ಉತ್ತರಿಸಿದ ಒಟ್ಟು ಮಕ್ಕಳ ಸಂಖ್ಯೆ}} & \textbf{\kn{ಶೇಕಡಾವಾರು ಫಲಿತಾಂಶ}} \\ \hline \endhead

{{loop.index}} & \kn{ {{ question.lang_name }} }& {{ question.num_correct }} & {{question.percent}} \\ \hline

\end{longtable}
}
\endgroup

\begin{flushleft}
\kn{ಈ ಮೇಲ್ಕಂಡ ಫಲಿತಾಂಶಗಳು ಇನ್ನೂ ಉತ್ತಮ ಪಡಿಸಲು ಶಾಲಾ ಮುಖ್ಯಗುರುಗಳು ,} SDMC \kn{, ಪೋಷಕರು ಮತ್ತು ಶಾಲಾ ಶಿಕ್ಷಕರ ಜೊತೆಗೂಡಿ ಸೂಕ್ತ ಯೋಜನೆಯನ್ನು ರೂಪಿಸುತ್ತೀರಾ ಎಂದು ಭಾವಿಸುತ್ತೇವೆ.} \\[2ex]
\end{flushleft}
\centering{\kn{ಸಲಹೆ}}\\



\ifnum {{loop.index}} = 1
    \begin{tabular}[t]{ll}
    \raisebox{-1.2cm}{\includegraphicsifexists[height=1.5cm]{"{{info.imagesqrdir}}{{num.competency}}_qrcode.png"}}  &
	    \fcolorbox{black}{gray}{\parbox[t][1.5cm]{4.5cm}{\centering{\kn{ {{num.local_name}} }\kn{ಪರಿಕಲ್ಪನೆಯ ಫಲಿತಾಂಶ ಮಕ್ಕಳಲ್ಲಿ  ಇನ್ನು ಉತ್ತಮ ಪಡಿಸಲು ಈ} QR \kn{ಕೋಡ್ ಅನ್ನು ಸ್ಕಾನ್ ಮಾಡಿ}}}}
    \end{tabular}
 \else 
     \ifnum {{loop.index}} = 2
     \hfill 
     \begin{tabular}[t]{ll}
     \raisebox{-1.2cm}{\includegraphicsifexists[height=1.5cm]{"{{info.imagesqrdir}}{{num.competency}}_qrcode.png"}} &
	     \fcolorbox{black}{lightgray}{\parbox[t][1.5cm]{4.5cm}{\centering{\kn{ {{num.local_name}} }\kn{ಪರಿಕಲ್ಪನೆಯ ಫಲಿತಾಂಶ ಮಕ್ಕಳಲ್ಲಿ  ಇನ್ನು ಉತ್ತಮ ಪಡಿಸಲು ಈ} QR \kn{ಕೋಡ್ ಅನ್ನು ಸ್ಕಾನ್ ಮಾಡಿ}}}}
     \end{tabular}
     \else
     \begin{center} 
     \begin{tabular}[t]{ll}
     \raisebox{-1.2cm}{\includegraphicsifexists[height=1.5cm]{"{{info.imagesqrdir}}{{num.competency}}_qrcode.png"}} &
	     \fbox{\parbox[t][1.5cm]{4.5cm}{\centering{\kn{ {{num.local_name}} }\kn{ಪರಿಕಲ್ಪನೆಯ ಫಲಿತಾಂಶ ಮಕ್ಕಳಲ್ಲಿ  ಇನ್ನು ಉತ್ತಮ ಪಡಿಸಲು ಈ} QR \kn{ಕೋಡ್ ಅನ್ನು ಸ್ಕಾನ್ ಮಾಡಿ}}}}
     \end{tabular}
     \end{center}
     \fi
\fi

	\ifnum {{assessmentinfo.deficiencies|length}} < 3
    ~\\~\\~\\~\\~\\
\else
    ~\\
\fi
\kn{ಸಹಿ} \hfill \kn{ಸಹಿ} \rule{3.8cm}{0cm} \\
\kn{ಜಿಲ್ಲಾ ಕ್ಷೇತ್ರ ವ್ಯವಸ್ಥಾಪಕರು, ಅಕ್ಷರ ಫೌಂಡೇಶನ್} \hfill \kn{ಅಧ್ಯಕ್ಷರು , ಗ್ರಾಮ ಪಂಚಾಯ್ತಿ} \rule{0.5cm}{0cm}
}

\end{document}
