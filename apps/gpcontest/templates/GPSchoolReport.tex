\documentclass[12pt]{article}
\usepackage[a4paper,top=2cm,left=2cm,right=2cm,bottom=0cm,portrait]{geometry}
\usepackage{longtable}
\usepackage{graphicx}
\usepackage{makecell}
\usepackage{lastpage}
\usepackage{multirow}
\usepackage{xcolor}
\usepackage{colortbl}
\usepackage{fancyhdr}
\pagestyle{fancy}
\usepackage[T1]{fontenc}
\usepackage{textcomp}
\usepackage{fontenc}
\usepackage{xltxtra}
\usepackage{tikz}


\newcommand{\kn}[1]{
{\fontspec{Kedage}
#1
}}

\fancypagestyle{plain}{
\fancyhf{} % clear all header and footer fields
\begin{tikzpicture}[remember picture,overlay]
\node[anchor=south west,yshift=0pt,xshift=0pt]%
at (current page.south west)
{\includegraphics[height=2cm]{"{{info.imagesdir}}footer_left.png"}};

\node[anchor=south east,yshift=0pt,xshift=0pt]%
at (current page.south east)
{\includegraphics[height=2cm]{"{{info.imagesdir}}footer_right.png"}};

\node[anchor=south west,yshift=5pt,xshift=100pt]%
at (current page.south west)
{\includegraphics[height=0.5cm]{"{{info.imagesdir}}footer_text.png"}};

\node[anchor=south east,yshift=5pt,xshift=-60pt]%
at (current page.south east)
{\includegraphics[height=1cm]{"{{info.imagesdir}}footer_logo.png"}};

\end{tikzpicture}

}
\renewcommand{\headrulewidth}{0pt} % to remove line on header
\renewcommand{\footrulewidth}{0pt} % to remove line on footer

\begin{document}

\thispagestyle{plain}
{
\setlength{\parindent}{0in}

\kn{ಜಿಲ್ಲೆ-} : {{schoolinfo.district}} \hfill  GP Id : {{schoolinfo.gpid}} \\
\kn{ತಾಲೂಕು} : {{schoolinfo.block}} \hfill DISE Id : {{schoolinfo.disecode}} \\
\kn{ಗ್ರಾಮ ಪಂಚಾಯ್ತಿ-} : {{schoolinfo.gpname}} \hfill \kn{ಶಾಲಾ ಹೆಸರು-} : {{schoolinfo.schoolname}}\\~\\

\kn{ಗ್ರಾಮ ಪಂಚಾಯ್ತಿ , ಶಿಕ್ಷಣ ಇಲಾಖೆ, ಸಮುದಾಯ ಮತ್ತು ಅಕ್ಷರ ಫೌಂಡೇಶನ್ ಸಹಭಾಗಿತ್ವದಲ್ಲಿ ದಿನಾಂಕ} DATE \kn{ರಂದು ನಡೆದ ಗ್ರಾಮ ಪಂಚಾಯ್ತಿ ಮಟ್ಟದಲ್ಲಿ ನಿಮ್ಮ ಶಾಲಾ ಮಕ್ಕಳ ಗಣಿತ ಸ್ಪರ್ಧೆಯ ತರಗತಿ ವಾರು ಕಿರು ವರದಿ ಈ ಕೆಳಕಂಡಂತಿದೆ}

\begin{longtable}{|p{1cm}|p{3cm}|p{3cm}|p{3cm}|} \hline
\multicolumn{4}{|c|}{4 \kn{ನೇ ತರಗತಿ}} \\ \hline
\multicolumn{4}{|c|}{\kn{ಮಕ್ಕಳ ಒಟ್ಟು ಸಂಖ್ಯೆ}} \\ \hline
\kn{ಕ್ರಮ ಸಂಖ್ಯೆ }&\kn{ಪ್ರಶ್ನೆಗಳು} & \kn{ಸರಿಯಾಗಿ ಉತ್ತರಿಸಿದ ಒಟ್ಟು  ಮಕ್ಕಳ ಸಂಖ್ಯೆ} & \kn{ಶೇಕಡಾವಾರು ಫಲಿತಾಂಶ } \\ \hline \endhead

{{loop.index}} & {{ question.text }} & {{ question.num_correct }} & {{question.percent}} \\ \hline


\end{longtable}

\kn{ಈ ಮೇಲ್ಕಂಡ ಫಲಿತಾಂಶಗಳು ಇನ್ನೂ ಉತ್ತಮ ಪಡಿಸಲು ಶಾಲಾ ಮುಖ್ಯಗುರುಗಳು , SDMC , ಪೋಷಕರು ಮತ್ತು ಶಾಲಾ ಶಿಕ್ಷಕರ ಜೊತೆಗೂಡಿ ಸೂಕ್ತ }\\
\centering{\kn{ಯೋಜನೆಯನ್ನು ರೂಪಿಸುತ್ತೀರಾ ಎಂದು}}\hfill \kn{ಭಾವಿಸುತ್ತೇವೆ}\\
\hfill\kn{ಸಲಹೆ}\\
}

\fbox{\parbox{4cm}{Checking if this works yes it does now does it?}}\hfill \fbox{\parbox{4cm}{Right side box second line}}\\
\centering{\fbox{\parbox[center]{4cm}{center text second line}}}
\pagebreak

%% for class 5
%%\thispagestyle{plain}
%%{

%%\setlength{\parindent}{0in}

%%\kn{ಜಿಲ್ಲೆ-} : {{schoolinfo.district}} \hfill  GP Id : {{schoolinfo.gpid}} \\
%\kn{ತಾಲೂಕು} : {{schoolinfo.block}} \hfill DISE Id : {{schoolinfo.disecode}} \\
%\kn{ಗ್ರಾಮ ಪಂಚಾಯ್ತಿ-} : {{schoolinfo.gpname}} \hfill \kn{ಶಾಲಾ ಹೆಸರು-} : {{schoolinfo.schoolname}}\\~\\

%\kn{}

%\begin{longtable}{|p{1cm}|p{3cm}|p{3cm}|p{3cm}|} \hline
%\multicolumn{4}{|c|}{4 \kn{}} \\ \hline
%\multicolumn{4}{|c|}{\kn{} \\ \hline
%\kn{}&\kn{} & \kn{} & \kn{} \\ \hline \endhead
%
%{loop.index}} & {{ question.text }} & {{ question.num_correct }} & {{question.percent}} \\ \hline
%

%\end{longtable}

%\kn{}\\
%\centering{\kn{}}\hfill \kn{}\\
%\hfill\kn{}\\
	
%}

%\pagebreak
%% for class 6
%\thispagestyle{plain}
%{

%\setlength{\parindent}{0in}

%\kn{ಜಿಲ್ಲೆ-} : {{schoolinfo.district}} \hfill  GP Id : {{schoolinfo.gpid}} \\
%\kn{ತಾಲೂಕು} : {{schoolinfo.block}} \hfill DISE Id : {{schoolinfo.disecode}} \\
%\kn{ಗ್ರಾಮ ಪಂಚಾಯ್ತಿ-} : {{schoolinfo.gpname}} \hfill \kn{ಶಾಲಾ ಹೆಸರು-} : {{schoolinfo.schoolname}}\\~\\

%\kn{}

%\begin{longtable}{|p{1cm}|p{3cm}|p{3cm}|p{3cm}|} \hline
%\multicolumn{4}{|c|}{4 \kn{}} \\ \hline
%\multicolumn{4}{|c|}{\kn{} \\ \hline
%\kn{}&\kn{} & \kn{} & \kn{} \\ \hline \endhead
%
%{loop.index}} & {{ question.text }} & {{ question.num_correct }} & {{question.percent}} \\ \hline
%

%\end{longtable}

%\kn{}\\
%\centering{\kn{}}\hfill \kn{}\\
%\hfill\kn{}\\
	
%}

\end{document}