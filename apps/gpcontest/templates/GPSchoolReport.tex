\documentclass[10pt]{article}
\usepackage[a4paper,top=1.5cm,left=2cm,right=2cm,bottom=0cm,portrait]{geometry}
\usepackage{longtable}
\usepackage{graphicx}
\usepackage{makecell}
\usepackage{lastpage}
\usepackage{multirow}
\usepackage{xcolor}
\usepackage{colortbl}
\usepackage{fancyhdr}
\pagestyle{fancy}
\usepackage[T1]{fontenc}
\usepackage{textcomp}
\usepackage{fontenc}
\usepackage{xltxtra}
\usepackage{tikz}
\usepackage{array}

\newcommand{\kn}[1]{
{\fontspec{Kedage}
#1
}}

\newcolumntype{P}[1]{>{\centering\arraybackslash}p{#1}}

\fancypagestyle{plain}{
\fancyhf{} % clear all header and footer fields
\lfoot{
\begin{tikzpicture}[remember picture,overlay]
\node[anchor=south west,yshift=0pt,xshift=0pt]%
at (current page.south west)
{\includegraphics[height=2cm]{"{{info.imagesdir}}footer_left.png"}};
\end{tikzpicture}}

\rfoot{
\begin{tikzpicture}[remember picture,overlay]
\node[anchor=south east,yshift=0pt,xshift=0pt]%
at (current page.south east)
{\includegraphics[height=2cm]{"{{info.imagesdir}}footer_right.png"}};
\end{tikzpicture}}

\cfoot{
\begin{tikzpicture}[remember picture,overlay]
\node[anchor=south west,yshift=7pt,xshift=4.5cm]%
at (current page.south west)
{\kn{ಗ್ರಾಮ ಪಂಚಾಯತಿ ಮಟ್ಟದ ಗಣಿತ ಸ್ಪರ್ಧೆಯ ವರದಿ} {{info.year}}};
\node[anchor=south,yshift=0pt,xshift=6cm]%
at (current page.south)
{\includegraphics[height=1cm]{"{{info.imagesdir}}footer_logo.png"}};
\end{tikzpicture}}
}

\renewcommand{\headrulewidth}{0pt} % to remove line on header
\renewcommand{\footrulewidth}{0pt} % to remove line on footer

\begin{document}

\thispagestyle{plain}
{
\setlength{\parindent}{0in}

\kn{ಜಿಲ್ಲೆ} : {{schoolinfo.district}} \hfill  GP Id : {{schoolinfo.gpid}} \\
\kn{ತಾಲೂಕು} : {{schoolinfo.block}} \hfill DISE Code : {{schoolinfo.disecode}} \\
\kn{ಗ್ರಾಮ ಪಂಚಾಯ್ತಿ} : {{schoolinfo.gpname}} \hfill \kn{ಶಾಲಾ ಹೆಸರು} : {{schoolinfo.schoolname}}\\

\kn{ಗ್ರಾಮ ಪಂಚಾಯ್ತಿ , ಶಿಕ್ಷಣ ಇಲಾಖೆ, ಸಮುದಾಯ ಮತ್ತು ಅಕ್ಷರ ಫೌಂಡೇಶನ್ ಸಹಭಾಗಿತ್ವದಲ್ಲಿ ದಿನಾಂಕ} {{schoolinfo.contestdate}} \kn{ರಂದು ಗ್ರಾಮ ಪಂಚಾಯ್ತಿ ಮಟ್ಟದಲ್ಲಿ ನಡೆದ ನಿಮ್ಮ ಶಾಲಾ ಮಕ್ಕಳ ಗಣಿತ ಸ್ಪರ್ಧೆಯ ತರಗತಿ ವಾರು ಕಿರು ವರದಿ ಈ ಕೆಳಕಂಡಂತಿದೆ}
\renewcommand\arraystretch{1.45}
\begin{longtable}{|P{1cm}|p{9cm}|P{3cm}|P{3cm}|} \hline
\multicolumn{4}{|c|}{ {\large \bf {{info.classname}} } \kn{ನೇ ತರಗತಿ}} \\ \hline
\multicolumn{4}{|c|}{\kn{ಮಕ್ಕಳ ಒಟ್ಟು ಸಂಖ್ಯೆ} : {{assessmentinfo.num_students}}} \\ \hline
\textbf{\kn{ಕ್ರಮ ಸಂಖ್ಯೆ}} & \textbf{\kn{ಪ್ರಶ್ನೆಗಳು}} & \textbf{\kn{ಸರಿಯಾಗಿ ಉತ್ತರಿಸಿದ ಒಟ್ಟು ಮಕ್ಕಳ ಸಂಖ್ಯೆ}} & \textbf{\kn{ಶೇಕಡಾವಾರು ಫಲಿತಾಂಶ}} \\ \hline \endhead

{{loop.index}} & \kn{ {{ question.lang_name }} }& {{ question.num_correct }} & {{question.percent}} \\ \hline


\end{longtable}

\kn{ಈ ಮೇಲ್ಕಂಡ ಫಲಿತಾಂಶಗಳು ಇನ್ನೂ ಉತ್ತಮ ಪಡಿಸಲು ಶಾಲಾ ಮುಖ್ಯಗುರುಗಳು ,} SDMC \kn{, ಪೋಷಕರು ಮತ್ತು ಶಾಲಾ ಶಿಕ್ಷಕರ ಜೊತೆಗೂಡಿ ಸೂಕ್ತ ಯೋಜನೆಯನ್ನು ರೂಪಿಸುತ್ತೀರಾ ಎಂದು ಭಾವಿಸುತ್ತೇವೆ.}\\
\centering{\kn{ಸಲಹೆ}}\\


\ifnum {{loop.index}} = 1
    \fcolorbox{black}{yellow}{\parbox{4cm}{\kn{ {{num.local_name}} }\kn{ಪರಿಕಲ್ಪನೆಯ ಫಲಿತಾಂಶ ಮಕ್ಕಳಲ್ಲಿ  ಇನ್ನು ಉತ್ತಮ ಪಡಿಸಲು ಈ} QR \kn{ಕೋಡ್ ಅನ್ನು ಸ್ಕಾನ್ ಮಾಡಿ}}}
 \else 
     \ifnum {{loop.index}} = 2
     \hfill \fcolorbox{black}{lightgray}{\parbox{4cm}{\kn{ {{num.local_name}} }\kn{ಪರಿಕಲ್ಪನೆಯ ಫಲಿತಾಂಶ ಮಕ್ಕಳಲ್ಲಿ  ಇನ್ನು ಉತ್ತಮ ಪಡಿಸಲು ಈ} QR \kn{ಕೋಡ್ ಅನ್ನು ಸ್ಕಾನ್ ಮಾಡಿ}}}
     \else
     \begin{center}\fbox{\parbox{4cm}{\kn{ {{num.local_name}} }\kn{ಪರಿಕಲ್ಪನೆಯ ಫಲಿತಾಂಶ ಮಕ್ಕಳಲ್ಲಿ  ಇನ್ನು ಉತ್ತಮ ಪಡಿಸಲು ಈ} QR \kn{ಕೋಡ್ ಅನ್ನು ಸ್ಕಾನ್ ಮಾಡಿ}}}\end{center}
     \fi
\fi

\kn{ಸಹಿ} \hfill \kn{ಸಹಿ} \\[3ex]
\kn{ಜಿಲ್ಲಾ ಕ್ಷೇತ್ರ ವ್ಯವಸ್ಥಾಪಕರು, ಅಕ್ಷರ ಫೌಂಡೇಶನ್} \hfill \kn{ಅಧ್ಯಕ್ಷರು , ಗ್ರಾಮ ಪಂಚಾಯ್ತಿ}
}

\end{document}