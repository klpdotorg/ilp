\documentclass[12pt]{article}
\usepackage[a4paper,top=2cm,left=2cm,right=2cm,bottom=0cm,portrait]{geometry}
\usepackage{longtable}
\usepackage{graphicx}
\usepackage{makecell}
\usepackage{lastpage}
\usepackage{multirow}
\usepackage{xcolor}
\usepackage{colortbl}
\usepackage{fancyhdr}
\pagestyle{fancy}
\usepackage[T1]{fontenc}
\usepackage{textcomp}
\usepackage{fontenc}
\usepackage{xltxtra}
\usepackage{tikz}


\newcommand{\kn}[1]{
{\fontspec{Kedage}
#1
}}

\fancypagestyle{plain}{
\fancyhf{} % clear all header and footer fields
\begin{tikzpicture}[remember picture,overlay]
\node[anchor=south west,yshift=0pt,xshift=0pt]%
at (current page.south west)
{\includegraphics[height=2cm]{"{{info.imagesdir}}footer_left.png"}};

\node[anchor=south east,yshift=0pt,xshift=0pt]%
at (current page.south east)
{\includegraphics[height=2cm]{"{{info.imagesdir}}footer_right.png"}};

\node[anchor=south west,yshift=5pt,xshift=100pt]%
at (current page.south west)
{\includegraphics[height=0.5cm]{"{{info.imagesdir}}footer_text.png"}};

\node[anchor=south east,yshift=5pt,xshift=-60pt]%
at (current page.south east)
{\includegraphics[height=1cm]{"{{info.imagesdir}}footer_logo.png"}};

\end{tikzpicture}

}

%% Headers and footers
\fancypagestyle{title}{
\fancyhf{} % clear all header and footer fields
\begin{tikzpicture}[remember picture,overlay]

\node[anchor=north, yshift=70pt, xshift=8.0cm]
{\includegraphics[width=3.5cm]{"{{info.imagesdir}}title_header_center.png"}};

\node[anchor=north west,yshift=0pt,xshift=0pt]%
at (current page.north west)
{\includegraphics[width=5cm]{"{{info.imagesdir}}title_header_left.png"}};

\node[anchor=north east,yshift=0pt,xshift=0pt]%
at (current page.north east)
{\includegraphics[width=3cm]{"{{info.imagesdir}}title_header_right.png"}};

\node[anchor=north west, yshift=-5cm, xshift=5cm]
{\includegraphics[width=7cm]{"{{info.imagesdir}}title_text.png"}};

\node[anchor=south west, yshift=200pt,xshift=8.5cm]%
	at (current page.south west)
{\includegraphics[width=4cm]{"{{info.imagesdir}}logo.jpg"}};

\node[anchor=south west,yshift=0pt,xshift=0pt]%
at (current page.south west)
{\includegraphics[height=4cm]{"{{info.imagesdir}}title_footer_left.png"}};

\node[anchor=south east,yshift=0pt,xshift=0pt]%
at (current page.south east)
{\includegraphics[height=3cm]{"{{info.imagesdir}}title_footer_right.png"}};


\end{tikzpicture}

%%\setlength{\headheight}{220pt}
%%\setlength{\footskip}{120pt}
}


\renewcommand{\headrulewidth}{0pt} % to remove line on header
\renewcommand{\footrulewidth}{0pt} % to remove line on footer




\begin{document}

\begin{titlepage}
    \thispagestyle{title}
    \begin{center}
    \vspace*{8cm}
     \textbf{ \huge \kn{ಗ್ರಾಮ ಪಂಚಾಯತಿ ಮಟ್ಟದ}} \\~\\
        \huge\kn{ಗಣಿತ ಸ್ಪರ್ಧೆಯ ವರದಿ}\\
	    {{info.year}}
    \end{center}
\end{titlepage}
\pagebreak

\thispagestyle{plain}
{
\setlength{\parindent}{0in}
\kn{ಇವರಿಗೆ,}\\
\kn{ಮಾನ್ಯ ಅಧ್ಯಕ್ಷರು,}\\
{{gpinfo.gpname}} \kn{ ಗ್ರಾಮ ಪಂಚಾಯತಿ,}\\
{{gpinfo.block}} \kn{ ತಾಲೂಕು,}\\
{{gpinfo.district}}\kn{ ಜಿಲ್ಲೆ.
\\~\\
ಮಾನ್ಯರೆ,
\\~\\
ವಿಷಯ: ಗಣಿತ ಕಲಿಕಾ ಆಂದೋಲನದ ಪರವಾಗಿ }{{gpinfo.gpname}}\kn{ಗ್ರಾಮ ಪಂಚಾಯತಿ ಮತ್ತು ಸಮುದಾಯದವರಿಗೆ ಅಭಿನಂದನೆಗಳು.
\\~\\
ಗ್ರಾಮ ಪಂಚಾಯತಿ, ಸಮುದಾಯ ಮತ್ತು ಅಕ್ಷರ ಫೌಂಡೇಶನ್ ಸಹಭಾಗಿತ್ವದಲ್ಲಿ ದಿನಾಂಕ}{{info.date}} \kn{ರಂದು ಗ್ರಾಮ ಪಂಚಾಯತಿ ಮಟ್ಟದಲ್ಲಿ 4, 5 ಮತ್ತು 6ನೇ ತರಗತಿಯ ಶಾಲಾ ಮಕ್ಕಳ ಗಣಿತ ಸ್ಪರ್ಧೆಯನ್ನು ನಿಮ್ಮ ಅಧ್ಯಕ್ಷತೆಯಲ್ಲಿ ಯಶಸ್ವಿಯಾಗಿ ನಡೆಸಿರುವುದಕ್ಕೆ ಅಭಿನಂದನೆಗಳನ್ನು ಸಲ್ಲಿಸುತ್ತೇವೆ. ಈ ಸ್ಪರ್ಧೆಯನ್ನು ಆಯೋಜಿಸುವ ಮೂಲಕ ಗ್ರಾಮ ಪಂಚಾಯ್ತಿಯು ಸಮುದಾಯದಲ್ಲಿ ಮಕ್ಳಳ ಕಲಿಕಾ ಗುಣಮಟ್ಟ ಹೆಚ್ಚಿಸಲು ಒಂದು ಮಹತ್ವವಾದ ಹೆಜ್ಜೆಯನ್ನು ಇಟ್ಟಿದೆ.}
\\~\\
\kn{ಗಣಿತ ಕಲಿಕಾ ಆಂದೋಲನದ ಈ ಗಣಿತ ಸ್ಪರ್ಧೆಯಲ್ಲಿ ತಮ್ಮ ಗ್ರಾಮ ಪಂಚಾಯತಿ ವ್ಯಾಪ್ತಿಯ ಎಲ್ಲಾ ಗ್ರಾಮಗಳಿಂದ ಒಟ್ಟು}{{gpinfo.totalstudents}} \kn{ಮಕ್ಕಳು ಭಾಗವಹಿಸಿರುತ್ತಾರೆ. ಎಲ್ಲಾ ಗ್ರಾಮಗಳ ಗ್ರಾಮ ಪಂಚಾಯತಿ ಸದಸ್ಯರು, ಪೋಷಕರು,} SDMC \kn{ಅಧ್ಯಕ್ಷರು ಮತ್ತು ಸದಸ್ಯರು, ಸ್ವಯಂಸೇವಕರು, ಸಮುದಾಯದ ಮುಖಂಡರು, ಶಿಕ್ಷಕರು ಹಾಗೂ ಸ್ಥಳೀಯ ಸಂಘ-ಸಂಸ್ಥೆಗಳು ಶಾಲಾ ಮಕ್ಕಳ ಗಣಿತ ಸ್ಪರ್ಧೆಗೆ ಸಂಪನ್ಮೂಲ, ಸಹಕಾರ ಹಾಗೂ ಸಹಭಾಗಿತ್ವದ ರೂಪದಲ್ಲಿ ಭಾಗವಹಿಸಿ ಯಶಸ್ವಿಗೊಳಿಸಿರುವುದಕ್ಕೆ ಎಲ್ಲಾ ಭಾಗೀದಾರರಿಗೂ ಧನ್ಯವಾದಗಳು.
\\~\\
ಗ್ರಾಮ ಪಂಚಾಯತಿ ಅಧ್ಯಕ್ಷರ ನಾಯಕತ್ವದಲ್ಲಿ ಸ್ಥಳೀಯ ಶಾಲಾ ಮಕ್ಕಳ ಗುಣಾತ್ಮಕ ಶಿಕ್ಷಣಕ್ಕಾಗಿ ಇಂತಹ ಶೈಕ್ಷಣಿಕ ಕಾರ್ಯಕ್ರಮಗಳನ್ನು ನಿಯಮಿತವಾಗಿ ಮಾಡಬೇಕೆಂದು ಕೋರುತ್ತೇವೆ.
\\~\\
ಈ ಪತ್ರದೊಂದಿಗೆ ನಿಮ್ಮ ಗ್ರಾಮ ಪಂಚಾಯತಿ ಮಟ್ಟದಲ್ಲಿ ನಡೆದ ಶಾಲಾ ಮಕ್ಕಳ ಗಣಿತ ಸ್ಪರ್ಧೆಯ ಸಂಕ್ಷಿಪ್ತ ವರದಿಯನ್ನು ಮುಂದಿನ ಯೋಜನೆ ಮತ್ತು ಸೂಕ್ತ ಕ್ರಮಕ್ಕಾಗಿ ಲಗತ್ತಿಸಲಾಗಿದೆ.
\\~\\~\\
ವಂದನೆಗಳೊಂದಿಗೆ
\\~\\~\\~\\
ಅಶೋಕ್ ಕಾಮತ್,
\\
ಅಧ್ಯಕ್ಷರು , ಅಕ್ಷರ ಫೌಂಡೇಶನ್\\
ಬೆಂಗಳೂರು\\
}
}

\pagebreak



\thispagestyle{plain}
{
\setlength{\parindent}{0in}
\kn{ಗ್ರಾಮ ಪಂಚಾಯ್ತಿ} : {{gpinfo.gpname}} \hfill \kn{ಜಿಲ್ಲೆ} : {{gpinfo.district}} \\
\kn{ತಾಲೂಕ} : {{gpinfo.block}} \hfill \kn{ಒಟ್ಟು ಶಾಲೆಗಳ ಸಂಖ್ಯೆ} : {{gpinfo.school_count}}\\
\kn{ಗ್ರಾಮ ಪಂಚಾಯ್ತಿ , ಶಿಕ್ಷಣ ಇಲಾಖೆ, ಸಮುದಾಯ ಮತ್ತು ಅಕ್ಷರ ಫೌಂಡೇಶನ್ ಸಹಭಾಗಿತ್ವದಲ್ಲಿ ದಿನಾಂಕ} {{info.date}} \kn{ರಂದು ನಡೆದ ಗ್ರಾಮ ಪಂಚಾಯ್ತಿ ಮಟ್ಟದಲ್ಲಿ ೪, ೫ ಮತ್ತು ೬ ನೇ ತರಗತಿಯ ಶಾಲಾ ಮಕ್ಕಳ ಗಣಿತ ಸ್ಪರ್ಧೆಯ ಕಿರು ವರದಿ ಈ ಕೆಳಕಂಡಂತಿದೆ.
ಈ ಗ್ರಾಮ ಪಂಚಾಯ್ತಿ ವ್ಯಾಪ್ತಿಯಲ್ಲಿ ಸರ್ಕಾರೀ ಶಾಲೆಗಳಿಂದ ೪, ೫ ಮತ್ತು ೬ನೇ ತರಗತಿಯ ಮಕ್ಕಳು ಗ್ರಾಮ ಪಂಚಾಯ್ತಿ ಮಟ್ಟದ ಗಣಿತ ಸ್ಪರ್ಧೆಯಲ್ಲಿ ಭಾಗವಹಿಸಿದ್ದರು.}
\\~\\
\kn{ತರಗತಿವಾರು ಗಣಿತ ಸ್ಪರ್ಧೆಯಲ್ಲಿ ಭಾಗವಹಿಸಿದ್ದ ಮಕ್ಕಳ ಒಟ್ಟಾರೆ ಫಲಿತಾಂಶಗಳು ಈ ಕೆಳಗಿನಂತೆ ಇವೆ.}\\

\begin{longtable}{|p{3cm}|p{2cm}|p{2cm}|p{2cm}|p{2cm}|p{2cm}|} \hline

\textbf{\kn{ತರಗತಿ}} & \textbf{\makecell[b]{\kn{ಒಟ್ಟು ಮಕ್ಕಳ ಸಂಖ್ಯೆ} & \multicolumn{4}{|c|}{\kn{ಅಂಕ ಗಳಿಸಿದ ಮಕ್ಕಳ ಸಂಖ್ಯೆ (ಶೇಕಡವಾರು)}}\\ \cline{3-6}
& &  >35\% & 36\%-60\%  & 61\%-75\% & 76\%-100\% \\ \hline \endhead
4\kn{ನೇ ತರಗತಿ} & {{class4.overall_scores.total}} & {{class4.overall_scores.below35}} & {{class4.overall_scores["35to60"]}} & {{class4.overall_scores["60to75"]}} & {{class4.overall_scores["75to100"]}} \\ \hline
5\kn{ನೇ ತರಗತಿ} & {{class5.overall_scores.total}} & {{class5.overall_scores.below35}} & {{class5.overall_scores["35to60"]}} & {{class5.overall_scores["60to75"]}} & {{class5.overall_scores["75to100"]}}   \\ \hline
6\kn{ನೇ ತರಗತಿ} & {{class6.overall_scores.total}} & {{class6.overall_scores.below35}} & {{class6.overall_scores["35to60"]}} & {{class6.overall_scores["60to75"]}} & {{class6.overall_scores["75to100"]}} \\ \hline

\end{longtable}

\begin{longtable}{|l|l|l|l|l|l|l|l|} \hline
\textbf{\kn{ತರಗತಿ}} & \textbf{\kn{ಒಟ್ಟು ಮಕ್ಕಳ ಸಂಖ್ಯೆ}} & \multicolumn{6}{|c|}{\textbf{\kn{ಮೂಲ ಕ್ರಿಯೆಗಳನ್ನು ಸರಿಯಾಗಿ ಮಾಡಬಲ್ಲ ಮಕ್ಕಳು}}} \\ \cline{3-8}
&& \kn{ಸಂಖ್ಯಾ ಪರಿಚಯ} & \kn{ಸ್ಥಾನ ಬೆಲೆ} & \kn{ಸಂಕಲನ} & \kn{ವ್ಯವಕಲನ} & \kn{ಗುಣಾಕಾರ} & \kn{ಭಾಗಾಕಾರ} \\ \hline \endhead
4\kn{ನೇ ತರಗತಿ} & {{class4.competency_scores.total}} & {{class4.competency_scores["Number Recognition"]}}  & {{class4.competency_scores["Place Value"]}} & {{class4.competency_scores["Addition"]}} & {{class4.competency_scores["Subraction"]}} & {{class4.competency_scores["Multiplication"]}} & {{class4.competency_scores["Division"]}} \\ \hline
5\kn{ನೇ ತರಗತಿ} & {{class5.competency_scores.total}} & {{class5.competency_scores["Number Recognition"]}}  & {{class5.competency_scores["Place Value"]}} & {{class5.competency_scores["Addition"]}} & {{class5.competency_scores["Subraction"]}} & {{class5.competency_scores["Multiplication"]}} & {{class5.competency_scores["Division"]}} \\ \hline
6\kn{ನೇ ತರಗತಿ} & {{class6.competency_scores.total}} & {{class6.competency_scores["Number Recognition"]}}  & {{class6.competency_scores["Place Value"]}} & {{class6.competency_scores["Addition"]}} & {{class6.competency_scores["Subraction"]}} & {{class6.competency_scores["Multiplication"]}} & {{class6.competency_scores["Division"]}} \\ \hline
\end{longtable}

 
\kn{ಫಲಿತಾಂಶ : ಒಟ್ಟಾರೆಯಾಗಿ ೫ ವರ್ಷ ಶಾಲೆಯಲ್ಲಿ ಕಲಿತ ಮಗು ಈ ಕೆಳಗಿನ ಮೂಲಕ್ರಿಯೆಯನ್ನು ಮಾಡಬಹುದು ಎಂದು ನಾವು ಪರಿಗಣಿಸಬಹುದು (ಶೇಕಡವಾರು)} \\
 

\begin{longtable}{|p{2cm}|p{2cm}|p{2cm}|p{2cm}|p{2cm}|p{2cm}|} \hline
\textbf{\kn{ಸಂಖ್ಯಾ ಪರಿಚಯವನ್ನು ಸರಿಯಾಗಿ ಮಾಡಬಲ್ಲ ಮಕ್ಕಳು}} & \textbf{\kn{ಸ್ಥಾನ ಬೆಲೆಯನ್ನು ಸರಿಯಾಗಿ ಮಾಡಬಲ್ಲ ಮಕ್ಕಳು}} & \textbf{\kn{ಸಂಕಲನವನ್ನು ಸರಿಯಾಗಿ ಮಾಡಬಲ್ಲ ಮಕ್ಕಳು}} & \textbf{\kn{ವ್ಯವಕಲನವನ್ನು ಸರಿಯಾಗಿ ಮಾಡಬಲ್ಲ ಮಕ್ಕಳು}} & \textbf{\kn{ಗುಣಾಕಾರವನ್ನು ಸರಿಯಾಗಿ ಮಾಡಬಲ್ಲ ಮಕ್ಕಳು}} & \textbf{\kn{ಭಾಗಾಕಾರವನ್ನು ಸರಿಯಾಗಿ ಮಾಡಬಲ್ಲ ಮಕ್ಕಳು}} \\ \hline \endhead
{{class6.percent_scores["Number Recognition"]}}&  {{class6.percent_scores["Place Value"]}} & {{class6.percent_scores["Addition"]}} & {{class6.percent_scores["Subtraction"]}} & {{class6.percent_scores["Multiplication"]}} &  {{class6.percent_scores["Division"]}} \\ \hline
\end{longtable}

\kn{ಈ ಮೇಲ್ಕಂಡ ಫಲಿತಾಂಶಗಳು ಇನ್ನೂ ಉತ್ತಮ ಪಡಿಸಲು ಗ್ರಾಮ ಪಂಚಾಯ್ತಿ ಜೊತೆಗೆ ಇತರೆ ಭಾಗೀದಾರರಾದ ನಾಗರಿಕ ಸೌಕರ್ಯ ಸಮಿತಿ}(CIVIL Amenities Committee), SDMC, \kn{ಪೋಷಕರು, ಶೈಕ್ಷಣಿಕ ಸ್ವಯಂಸೇವಕರು ಮತ್ತು ಶಾಲಾ ಶಿಕ್ಷಕರ ಜೊತೆಗೂಡಿ ಯೋಜನೆಯನ್ನು ರೂಪಿಸುತ್ತೀರಾ ಎಂದು ಭಾವಿಸುತ್ತೇವೆ.} \break

\centering{\kn{ವಂದನೆಗಳೊಂದಿಗೆ}} 
\\~\\~\\~\\
\kn{ಜೆ.ವಿ.ಶಂಕರನಾರಾಯಣ} \hfill \kn{ನಾಗರಾಜ್ ಪ್ರಭು}

\kn{ಮುಖ್ಯಸ್ಥರು,ಕಾರ್ಯಾಚರಣೆ ಮತ್ತು ಸಮುದಾಯ ಅಭಿವೃದ್ಧಿ} \hfill \kn{ಮುಖ್ಯಸ್ಥರು,ಗಣಿತ ಸಂಪನ್ಮೂಲ ಮತ್ತು} GKA

\kn{ಅಕ್ಷರ ಫೌಂಡೇಶನ್,ಬೆಂಗಳೂರು.} \hfill \kn{ಅಕ್ಷರ ಫೌಂಡೇಶನ್,ಬೆಂಗಳೂರು.}
}

\end{document}
