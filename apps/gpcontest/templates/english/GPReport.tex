\documentclass[12pt]{article}
\usepackage[a4paper,top=2cm,left=1cm,right=1cm,bottom=2cm,portrait]{geometry}
\usepackage{longtable}
\usepackage{graphicx}
\usepackage{makecell}
\usepackage{multirow}
\usepackage[table]{xcolor}
\usepackage{colortbl}
\usepackage{fancyhdr}
\usepackage[T1]{fontenc}
\usepackage{textcomp}
\usepackage{fontenc}
\usepackage{xltxtra}
\usepackage{tikz}
\usepackage{array}
\usepackage[absolute,overlay]{textpos}


\newcolumntype{P}[1]{>{\centering\arraybackslash}p{#1}}

\fancypagestyle{plain}{
\fancyhf{} % clear all header and footer fields
\lfoot{
\begin{tikzpicture}[overlay]
\node[anchor=south west,yshift=-1.1cm,xshift=-1.2cm]
{\includegraphics[height=2cm]{"{{info.imagesdir}}footer_left.png"}};
\end{tikzpicture}}

\rfoot{
\begin{tikzpicture}[overlay]
\node[anchor=south east,yshift=-1.1cm,xshift=1.2cm]
{\includegraphics[height=2cm]{"{{info.imagesdir}}footer_right.png"}};
\end{tikzpicture}}

\cfoot{
\fontsize{9}{12}
\begin{tikzpicture}[overlay]
\node[anchor=south west,yshift=0cm,xshift=-6cm]
{For more information on the GP level Math contest report, {{info.acadyear}} };
\node[anchor=south west,yshift=-0.5cm,xshift=-6cm]
{Please call - 9845079590  };
\node[anchor=south,yshift=-0.5cm,xshift=6cm]
{\includegraphics[height=1cm]{"{{info.imagesdir}}footer_logo.png"}};
\end{tikzpicture}}
}

%% Headers and footers
\fancypagestyle{title}{
\fancyhf{} % clear all header and footer fields
\chead{
\begin{tikzpicture}[overlay]
\node[anchor=north, yshift=1.7cm, xshift=0cm]
{\includegraphics[width=3.5cm]{"{{info.imagesdir}}title_header_center.png"}};
\end{tikzpicture}}

\lhead{
\begin{tikzpicture}[overlay]
\node[anchor=north west,yshift=1.7cm,xshift=-1.2cm]
{\includegraphics[width=5cm]{"{{info.imagesdir}}title_header_left.png"}};
\end{tikzpicture}}

\rhead{
\begin{tikzpicture}[overlay]
\node[anchor=north east,yshift=1.7cm,xshift=1.2cm]
{\includegraphics[width=3cm]{"{{info.imagesdir}}title_header_right.png"}};
\end{tikzpicture}}

\begin{tikzpicture}[overlay]
\node[anchor=north west, yshift=-6.5cm, xshift=5cm]
{\includegraphics[width=7cm]{"{{info.imagesdir}}title_text.png"}};
\end{tikzpicture}

\begin{tikzpicture}[overlay]
%\node[anchor=south west, yshift=200pt,xshift=5cm]
\node[anchor=north west, yshift=-16.5cm,xshift=6cm]
{\includegraphics[width=4cm]{"{{info.imagesdir}}logo.jpg"}};
\end{tikzpicture}

\lfoot{
\begin{tikzpicture}[overlay]
\node[anchor=south west,yshift=-0.6cm,xshift=-1.2cm]
{\includegraphics[height=4cm]{"{{info.imagesdir}}title_footer_left.png"}};
\end{tikzpicture}}

\rfoot{
\begin{tikzpicture}[overlay]
\node[anchor=south east,yshift=-0.6cm,xshift=1.2cm]
{\includegraphics[height=3cm]{"{{info.imagesdir}}title_footer_right.png"}};
\end{tikzpicture}}

\setlength{\headheight}{15pt}
%%\setlength{\footskip}{120pt}
}


\renewcommand{\headrulewidth}{0pt} % to remove line on header
\renewcommand{\footrulewidth}{0pt} % to remove line on footer




\begin{document}

\begin{titlepage}
    \thispagestyle{title}
    \begin{center}
    \vspace*{10.5cm}
     \textbf{\huge Gram Panchayat level} \\~\\
        \textbf{\huge Math contest Report} \\~\\
	    \Large{{info.acadyear}}
    \end{center}
\end{titlepage}
\pagebreak

\thispagestyle{plain}
{
\setlength{\parindent}{0in}
~\\~\\~\\
To, \hfill  {{info.month}} {{info.year}} \\ [2ex]
The GP Sarpanch,\\ [1ex]
{{gpinfo.gpname}} GP,\\ [1ex]
{{gpinfo.block}} Taluk,\\ [1ex]
{{gpinfo.district}} District.
\\~\\ [2ex]
Sir/Madam,
\\~\\
Subject: A hearty congratulations and thank you to the {{gpinfo.gpname}} Gram Panchayat on behalf of the Ganitha Kalika Andolana (GKA) team.
\\~\\
A hearty congratulations to the Gram Panchayat members, Community members and the respective school HMs for successfully conducting the Gram Panchayat level Math Contest for 4 th , 5 th and 6 th grade under the able leadership of the GP sarpanch on {{gpinfo.contestdate}}. By facilitating this contest in your Gram Panchayat, you have taken a major step towards improving learning outcomes among your children in your villages.
\\~\\
As part of the GKA GP level math contest, {{gpinfo.totalstudents}} students from all the schools in all villages under your GP have participated. We would like to thank all the stakeholders – GP members, Parents, SMC president and members, village volunteers, local leaders, local organization and children for making this a successful event who helped with their resources, cooperation and participation.
\\~\\
Under the leadership of the GP sarpanch, we request you conduct such events and programmes for quality education and better learning outcomes.
\\~\\
Attached along with this letter is the brief summary of the result of the GP level math contest. Please take the necessary and suitable action based on the school level result.
\\~\\~\\
Thank You,\\~\\~\\~\\
\\
\begin{tikzpicture}[overlay]
\node[anchor=south east,yshift=0.1cm,xshift=3.1cm]
{\includegraphics[width=3cm,height=1.3cm]{"{{info.imagesdir}}Ashoks_signature.png"}};
\end{tikzpicture}
\\
Ashok Kamath,\\
Chairman, Akshara Foundation\\
Bangalore\\
}

\pagebreak



\thispagestyle{plain}
{
\setlength{\parindent}{0in}

GP Name : {{gpinfo.gpname}} \hfill District : {{gpinfo.district}}\ \\ [1ex]
Taluk : {{gpinfo.block}}\hfill Total Schools : {{gpinfo.school_count}} \\~\\~\\~\\

~\\~\\

The following is the short report of the 4th ,5 th and 6 th grade math competition held at Gram Panchayat level on {{gpinfo.contestdate}} by the Gram Panchayat, Education Department in collaboration with the Community and Akshara Foundation.
\\~\\
A total of {{gpinfo.totalstudents}} children from classes 4 th , 5 th and 6 th from government schools in this grama panchayat area participated in the grama panchayat level mathematics competition.
\\~\\
Grade-wise report of the GP level math contest is as follows:\\
\begin{longtable}{|P{3cm}|P{4cm}|P{2.2cm}|P{2.2cm}|P{2.2cm}|P{2.2cm}|} \hline
\textbf{Grades} & \textbf{Total number of Children} & \multicolumn{4}{|c|}{\textbf{Number of children scored (percentage)}}\\ \cline{3-6}
\rule{0cm}{0.3cm}& &  \textbf{<35\%} & \textbf{35\%-59\%}  & \textbf{60\%-74\%} & \textbf{75\%-100\%} \\ \hline \endhead

\rule{0cm}{0.3cm} {{ assessment["class"] }} Grade & {{ assessment["overall_scores"]["total"]}} & {{assessment["overall_scores"]["below35"]}} & {{assessment["overall_scores"]["35to59"]}} & {{assessment["overall_scores"]["60to74"]}} & {{assessment["overall_scores"]["75to100"]}} \\ \hline

\end{longtable}

\begingroup
\fontsize{11}{12}
{
\setlength\LTleft{-0.1cm}
\setlength\LTright{-0.1cm}
\begin{longtable}{|P{2.5cm}|P{3.25cm}|c|c|c|c|c|c|} \hline
	\textbf{Grades} & \textbf{\makecell{Total number \\of Children}} & \multicolumn{6}{|c|}{\textbf{Average percentage marks scored by children in basic operation}} \\ \cline{3-8}
	\rule{0cm}{0.3cm} &&\textbf{\makecell{Number \\Sense}}&\textbf{\makecell{Place\\ Value}}&\textbf{Addition}&\textbf{Subtraction}&\textbf{Multiplication}&\textbf{Division} \\ \hline \endhead

\rule{0cm}{0.3cm} {{assessment["class"]}} Grade & {{assessment["competency_scores"]["total"]}} & {{assessment["competency_scores"]["Number Recognition"]}}  & {{assessment["competency_scores"]["Place Value"]}} & {{assessment["competency_scores"]["Addition"]}} & {{assessment["competency_scores"]["Subtraction"]}} & {{assessment["competency_scores"]["Multiplication"]}} & {{assessment["competency_scores"]["Division"]}} \\ \hline

\end{longtable}
}
\endgroup
 

{
~\\
\centering{Following are the overall percentage-wise results of children who have completed 5 years of school (Grade 6). \\[2ex]
 
\begingroup
\fontsize{8.75}{12}
{

{
\centering{
\begin{longtable}{|c|c|c|c|c|c|} \hline
\cellcolor{white}{\textbf{\makecell[b]{Children who can do \\number recognition correctly}}} & \cellcolor{white}{\textbf{\makecell[b]{Children who can do \\place value correctly}}} & \textbf{\makecell[b]{Children who can do \\addition correctly}} & \cellcolor{white}{\textbf{\makecell[b]{Subtraction}}} & \cellcolor{white}{\textbf{\makecell[b]{Children who can do \\multiplication correctly}}} & \textbf{\makecell[b]{Children who can do \\division correctly}} \\ \hline \endhead

\rule{0cm}{0.3cm} {{percent_scores["assessments"][assessment]["Number Recognition"]}}&  {{percent_scores["assessments"][assessment]["Place Value"]}} & {{percent_scores["assessments"][assessment]["Addition"]}} & {{percent_scores["assessments"][assessment]["Subtraction"]}} & {{percent_scores["assessments"][assessment]["Multiplication"]}} &  {{percent_scores["assessments"][assessment]["Division"]}}  \\ \hline

\end{longtable}
}
}

{
\centering{
\begin{longtable}{|c|c|c|c|} \hline
 \textbf{\makecell[b]{Children who can do \\addition correctly}} & \cellcolor{white}{\textbf{\makecell[b]{Children who can do \\subtraction correctly}}} & \cellcolor{white}{\textbf{\makecell[b]{Children who can do \\multiplication correctly}}} & \textbf{\makecell[b]{Children who can do \\division correctly}} \\ \hline \endhead

\rule{0cm}{0.3cm} {{percent_scores["assessments"][assessment]["Addition"]}} & {{percent_scores["assessments"][assessment]["Subtraction"]}} & {{percent_scores["assessments"][assessment]["Multiplication"]}} &  {{percent_scores["assessments"][assessment]["Division"]}}  \\ \hline
	
\end{longtable}
}
}


}
\endgroup
}


\begin{flushleft}
Note: Addition with decimal, subtraction with decimal, and division with remainder are considered for analysis in the tables of this report.\\
To improve the above results, we hope that the Gram Panchayat will collaborate with other stakeholders such as the CIVIL Amenities Committee, the SMC, parents, educational volunteers, and school teachers. \\[5ex]
\end{flushleft}

\centering{Thank You} \\[10ex]

\begin{flushleft}
From\\
Akshara Foundation
\end{flushleft}
}
\end{document}
