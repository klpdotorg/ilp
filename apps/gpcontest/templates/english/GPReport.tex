\documentclass[12pt]{article}
\usepackage[a4paper,top=2cm,left=1cm,right=1cm,bottom=2cm,portrait]{geometry}
\usepackage{longtable}
\usepackage{graphicx}
\usepackage{makecell}
\usepackage{multirow}
\usepackage[table]{xcolor}
\usepackage{colortbl}
\usepackage{fancyhdr}
\usepackage[T1]{fontenc}
\usepackage{textcomp}
\usepackage{fontenc}
\usepackage{xltxtra}
\usepackage{tikz}
\usepackage{array}
\usepackage[absolute,overlay]{textpos}


\newcolumntype{P}[1]{>{\centering\arraybackslash}p{#1}}

\fancypagestyle{plain}{
\fancyhf{} % clear all header and footer fields
\lfoot{
\begin{tikzpicture}[overlay]
\node[anchor=south west,yshift=-1cm,xshift=-1cm]
{\includegraphics[height=2cm]{"{{info.imagesdir}}footer_left.png"}};
\end{tikzpicture}}

\rfoot{
\begin{tikzpicture}[overlay]
\node[anchor=south east,yshift=-1cm,xshift=1cm]
{\includegraphics[height=2cm]{"{{info.imagesdir}}footer_right.png"}};
\end{tikzpicture}}

\cfoot{
\fontsize{9}{12}
\begin{tikzpicture}[overlay]
\node[anchor=south west,yshift=0cm,xshift=-6cm]
{For more information on the GP level Math contest report, {{info.acadyear}} };
\node[anchor=south west,yshift=-0.5cm,xshift=-6cm]
{Please call - 9845079590  };
\node[anchor=south,yshift=-0.5cm,xshift=6cm]
{\includegraphics[height=1cm]{"{{info.imagesdir}}footer_logo.png"}};
\end{tikzpicture}}
}

%% Headers and footers
\fancypagestyle{title}{
\fancyhf{} % clear all header and footer fields
\chead{
\begin{tikzpicture}[overlay]
\node[anchor=north, yshift=1.5cm, xshift=0cm]
{\includegraphics[width=3.5cm]{"{{info.imagesdir}}title_header_center.png"}};
\end{tikzpicture}}

\lhead{
\begin{tikzpicture}[overlay]
\node[anchor=north west,yshift=1.5cm,xshift=-1cm]
{\includegraphics[width=5cm]{"{{info.imagesdir}}title_header_left.png"}};
\end{tikzpicture}}

\rhead{
\begin{tikzpicture}[overlay]
\node[anchor=north east,yshift=1.5cm,xshift=1cm]
{\includegraphics[width=3cm]{"{{info.imagesdir}}title_header_right.png"}};
\end{tikzpicture}}

\begin{tikzpicture}[overlay]
\node[anchor=north west, yshift=-6.5cm, xshift=5cm]
{\includegraphics[width=7cm]{"{{info.imagesdir}}title_text.png"}};
\end{tikzpicture}

\begin{tikzpicture}[overlay]
%\node[anchor=south west, yshift=200pt,xshift=5cm]
\node[anchor=north west, yshift=-16.5cm,xshift=6cm]
{\includegraphics[width=4cm]{"{{info.imagesdir}}logo.jpg"}};
\end{tikzpicture}

\lfoot{
\begin{tikzpicture}[overlay]
\node[anchor=south west,yshift=0cm,xshift=-1cm]
{\includegraphics[height=4cm]{"{{info.imagesdir}}title_footer_left.png"}};
\end{tikzpicture}}

\rfoot{
\begin{tikzpicture}[overlay]
\node[anchor=south east,yshift=0cm,xshift=1cm]
{\includegraphics[height=3cm]{"{{info.imagesdir}}title_footer_right.png"}};
\end{tikzpicture}}

\setlength{\headheight}{15pt}
%%\setlength{\footskip}{120pt}
}


\renewcommand{\headrulewidth}{0pt} % to remove line on header
\renewcommand{\footrulewidth}{0pt} % to remove line on footer




\begin{document}

\begin{titlepage}
    \thispagestyle{title}
    \begin{center}
    \vspace*{10.5cm}
     \textbf{\huge Gram Panchayat level} \\~\\
        \textbf{\huge Math contest Report} \\~\\
	    \Large{{info.acadyear}}
    \end{center}
\end{titlepage}
\pagebreak

\thispagestyle{plain}
{
\setlength{\parindent}{0in}
~\\~\\~\\
To, \hfill  {{info.month}} {{info.year}} \\ [2ex]
The GP President,\\ [1ex]
{{gpinfo.gpname}} GP,\\ [1ex]
{{gpinfo.block}} Block,\\ [1ex]
{{gpinfo.district}} District.
\\~\\ [2ex]
Sir/Madame,
\\~\\
Subject: A hearty congratulations and thank you to {{gpinfo.gpname}} Grama Panchayati on behalf of Ganitha Kalika Andolana (GKA) team.
\\~\\
A hearty congratulations to the Grama Panchayat members, Community members and the respective school HMs for successfully conducting the Gram Panchayat level Math Contest under the able leadership of the GP president on {{gpinfo.contestdate}}. By facilitating this contest in your Gram Panchayat, you have taken a major step towards improving learning outcome among your children in your villages.
\\~\\
As part of the GKA GP level math contest, {{gpinfo.totalstudents}} students from all the schools in all villages under your GP have participated. We would like to thank all the stakeholders – GP members, Parents, SDMC president and members, village volunteers, local leaders, local organization and children for making this a successful event who helped with their resources, cooperation and participation.
\\~\\
Under the leadership of the GP president, we request you conduct such events and programmes for quality education and better learning outcome.
\\~\\
Attached along with this letter is the brief summary of the result of GP level math contest. Please take the necessary and suitable action based on the school level result.
\\~\\~\\
Thank You,
\\~\\~\\~\\
Ashok Kamath,\\
Chairman, Akshara Foundation\\
Bangalore\\
}

\pagebreak



\thispagestyle{plain}
{
\setlength{\parindent}{0in}

Gram Panchayat : {{gpinfo.gpname}} \hfill District : {{gpinfo.district}}\ \\ [1ex] 
Block : {{gpinfo.block}}\hfill Total Schools : {{gpinfo.school_count}} \\~\\~\\~\\

~\\~\\

A brief summary of the GP level math contest report that was held on {{gpinfo.contestdate}} for the students of grades 4, 5 and 6 from the government primary school under the {{gpinfo.gpname}} GP is as follows. The contest was held under the leadership of GP president and in partnership with education department, community and Akshara Foundation.
\\~\\
Grade-wise report of the GP level math contest is as follows:\\
\begin{longtable}{|P{3cm}|P{4cm}|P{2.2cm}|P{2.2cm}|P{2.2cm}|P{2.2cm}|} \hline
\textbf{Grades} & \textbf{Total Children} & \multicolumn{4}{|c|}{\textbf{Marks obtained (percentage)}}\\ \cline{3-6}
\rule{0cm}{0.3cm}& &  \textbf{<35\%} & \textbf{35\%-59\%}  & \textbf{60\%-74\%} & \textbf{75\%-100\%} \\ \hline \endhead

\rule{0cm}{0.3cm} {{ assessment["class"] }} Grade & {{ assessment["overall_scores"]["total"]}} & {{assessment["overall_scores"]["below35"]}} & {{assessment["overall_scores"]["35to59"]}} & {{assessment["overall_scores"]["60to74"]}} & {{assessment["overall_scores"]["75to100"]}} \\ \hline

\end{longtable}

\begingroup
\fontsize{11}{12}
{
\setlength\LTleft{-0.1cm}
\setlength\LTright{-0.1cm}
\begin{longtable}{|P{2.5cm}|P{3.25cm}|c|c|c|c|c|c|} \hline
\textbf{Grades} & \textbf{Total Children} & \multicolumn{6}{|c|}{\textbf{Children who can do basic arithmetic operations}} \\ \cline{3-8}
\rule{0cm}{0.3cm} &&\textbf{Number Sense}&\textbf{Place Value}&\textbf{Addition}&\textbf{Subtraction}&\textbf{Multiplication}&\textbf{Division} \\ \hline \endhead

\rule{0cm}{0.3cm} {{assessment["class"]}} Grade & {{assessment["competency_scores"]["total"]}} & {{assessment["competency_scores"]["Number Recognition"]}}  & {{assessment["competency_scores"]["Place Value"]}} & {{assessment["competency_scores"]["Addition"]}} & {{assessment["competency_scores"]["Subtraction"]}} & {{assessment["competency_scores"]["Multiplication"]}} & {{assessment["competency_scores"]["Division"]}} \\ \hline

\end{longtable}
}
\endgroup
 

{
~\\
\centering{Result: After completion of 5 years of study in the school, grade 6 students can do the following math operation (percentage)} \\[2ex]
 
\begingroup
\fontsize{8.75}{12}
{

{
\centering{
\begin{longtable}{|c|c|c|c|c|c|} \hline
\cellcolor{gray}{\textbf{\makecell[b]{Number\\Sense}}} & \cellcolor{lightgray}{\textbf{\makecell[b]{Place\\Value}}} & \textbf{\makecell[b]{Addition}} & \cellcolor{gray}{\textbf{\makecell[b]{Subtraction}}} & \cellcolor{lightgray}{\textbf{\makecell[b]{Multiplication}}} & \textbf{\makecell[b]{Division}} \\ \hline \endhead

\rule{0cm}{0.3cm} {{percent_scores["assessments"][assessment]["Number Recognition"]}}&  {{percent_scores["assessments"][assessment]["Place Value"]}} & {{percent_scores["assessments"][assessment]["Addition"]}} & {{percent_scores["assessments"][assessment]["Subtraction"]}} & {{percent_scores["assessments"][assessment]["Multiplication"]}} &  {{percent_scores["assessments"][assessment]["Division"]}}  \\ \hline

\end{longtable}
}
}

{
\centering{
\begin{longtable}{|c|c|c|c|} \hline
 \textbf{\makecell[b]{Addition}} & \cellcolor{gray}{\textbf{\makecell[b]{Subtraction}}} & \cellcolor{lightgray}{\textbf{\makecell[b]{Multiplication}}} & \textbf{\makecell[b]{Division}} \\ \hline \endhead

\rule{0cm}{0.3cm} {{percent_scores["assessments"][assessment]["Addition"]}} & {{percent_scores["assessments"][assessment]["Subtraction"]}} & {{percent_scores["assessments"][assessment]["Multiplication"]}} &  {{percent_scores["assessments"][assessment]["Division"]}}  \\ \hline
	
\end{longtable}
}
}


}
\endgroup
}


\begin{flushleft}
There is abundant scope to improve the above results for the teachers and parents. We hope that the schools can work with the Parents, Village education volunteers, Grama Panchayat, SDMC and Civil amenities committee to improve the learning outcome in their respective schools by drawing up a strategic plan for mathematics teaching. \\[5ex]
\end{flushleft}


\centering{Thank You} \\[10ex]

~\\~\\


\begin{flushleft}
Sign,\\
District Manager, Akshara Foundation
\end{flushleft}
}
\end{document}
