\documentclass[12pt]{article}
\usepackage[a4paper,top=2cm,left=1cm,right=1cm,bottom=2cm,portrait]{geometry}
\usepackage{longtable}
\usepackage{graphicx}
\usepackage{makecell}
\usepackage{multirow}
\usepackage[table]{xcolor}
\usepackage{colortbl}
\usepackage{fancyhdr}
\usepackage[T1]{fontenc}
\usepackage{textcomp}
\usepackage{fontenc}
\usepackage{xltxtra}
\usepackage{tikz}
\usepackage{array}
\usepackage[absolute,overlay]{textpos}


\newcolumntype{P}[1]{>{\centering\arraybackslash}p{#1}}

\fancypagestyle{plain}{
\fancyhf{} % clear all header and footer fields
\lfoot{
\begin{tikzpicture}[overlay]
\node[anchor=south west,yshift=-1cm,xshift=-1cm]
{\includegraphics[height=2cm]{"{{info.imagesdir}}footer_left.png"}};
\end{tikzpicture}}

\rfoot{
\begin{tikzpicture}[overlay]
\node[anchor=south east,yshift=-1cm,xshift=1cm]
{\includegraphics[height=2cm]{"{{info.imagesdir}}footer_right.png"}};
\end{tikzpicture}}

\cfoot{
\fontsize{9}{12}
\begin{tikzpicture}[overlay]
\node[anchor=south west,yshift=0cm,xshift=-6cm]
{For more information on the GP level Math contest report, {{info.acadyear}} };
\node[anchor=south west,yshift=-0.5cm,xshift=-6cm]
{Please call - 9845079590  };
\node[anchor=south,yshift=-0.5cm,xshift=6cm]
{\includegraphics[height=1cm]{"{{info.imagesdir}}footer_logo.png"}};
\end{tikzpicture}}
}

%% Headers and footers
\fancypagestyle{title}{
\fancyhf{} % clear all header and footer fields
\chead{
\begin{tikzpicture}[overlay]
\node[anchor=north, yshift=1.5cm, xshift=0cm]
{\includegraphics[width=3.5cm]{"{{info.imagesdir}}title_header_center.png"}};
\end{tikzpicture}}

\lhead{
\begin{tikzpicture}[overlay]
\node[anchor=north west,yshift=1.5cm,xshift=-1cm]
{\includegraphics[width=5cm]{"{{info.imagesdir}}title_header_left.png"}};
\end{tikzpicture}}

\rhead{
\begin{tikzpicture}[overlay]
\node[anchor=north east,yshift=1.5cm,xshift=1cm]
{\includegraphics[width=3cm]{"{{info.imagesdir}}title_header_right.png"}};
\end{tikzpicture}}

\begin{tikzpicture}[overlay]
\node[anchor=north west, yshift=-6.5cm, xshift=5cm]
{\includegraphics[width=7cm]{"{{info.imagesdir}}title_text.png"}};
\end{tikzpicture}

\begin{tikzpicture}[overlay]
%\node[anchor=south west, yshift=200pt,xshift=5cm]
\node[anchor=north west, yshift=-16.5cm,xshift=6cm]
{\includegraphics[width=4cm]{"{{info.imagesdir}}logo.jpg"}};
\end{tikzpicture}

\lfoot{
\begin{tikzpicture}[overlay]
\node[anchor=south west,yshift=0cm,xshift=-1cm]
{\includegraphics[height=4cm]{"{{info.imagesdir}}title_footer_left.png"}};
\end{tikzpicture}}

\rfoot{
\begin{tikzpicture}[overlay]
\node[anchor=south east,yshift=0cm,xshift=1cm]
{\includegraphics[height=3cm]{"{{info.imagesdir}}title_footer_right.png"}};
\end{tikzpicture}}

\setlength{\headheight}{15pt}
%%\setlength{\footskip}{120pt}
}

\renewcommand{\headrulewidth}{0pt} % to remove line on header
\renewcommand{\footrulewidth}{0pt} % to remove line on footer


\begin{document}

\begin{titlepage}
    \thispagestyle{title}
    \begin{center}
    \vspace*{10.5cm}
     \textbf{\huge Gram Panchayat level} \\~\\
        \textbf{\huge Math contest Report District level} \\~\\
	    \Large{{info.acadyear}}
    \end{center}
\end{titlepage}
\pagebreak

\thispagestyle{plain}
{
\setlength{\parindent}{0in}
To, \hfill  {{info.month}} {{info.year}} \\ [2ex]
{{sendto.langname}} ,\\ [1ex]
{{districtinfo.name}} District.
\\~\\ [3ex]
Dear Sir/Madam,
\\~\\
\textbf{Subject:} Report on the Gram Panchayat(GP) Level Government School Children Mathematics Competition held in {{districtinfo.name}} district during {{info.acadyear}} under Ganitha Kalika Andolana (GKA) in collaboration with the Rural Development and Panchayat Raj Department, the Community, the Department of School Education and Literacy, and Akshara Foundation.
\\~\\[3ex]
The Ganitha Kalika Andolana is a government programme run by the Department of School Education and Literacy. In collaboration with Akshara Foundation and community of village panchayats along with the Rural Development and Panchayat Raj Department, Department of School Education and Literacy in collaboration and  government schools in {{districtinfo.num_blocks}} taluks {{districtinfo.num_gps}} Grama panchayats in {{districtinfo.name}} district,  Gram Panchayat Level Government School Children Mathematics Competition was held for {{districtinfo.totalstudents}} children of {{districtinfo.num_schools}} Government schools  in the 4th ,5th , and 6th  grades during the academic year {{info.acadyear}}.
\\~\\[2ex]
All Sarpanch/Members of Panchayat in this district, Parents, SMC Presidents/ Members, Volunteers, village heads, headmasters /teachers and Members of the local association participated and made the Mathematics GP Competition a success by providing the needed  resources, cooperation and encouragement.
\\~\\[2ex]
We request that mathematics GP competitions be held regularly at the panchayat level under the direction of the president of the District grama panchayat in order to provide government school students with a high-quality education.
\\~\\
A consolidated report of the mathematics competition of all {{districtinfo.num_blocks}} taluk of all panchayats in their district is enclosed with this letter for further planning and appropriate action.
\\~\\~\\
Thank You,\\~\\~\\~\\
\\
\begin{tikzpicture}[overlay]
\node[anchor=south east,yshift=0.1cm,xshift=3.1cm]
{\includegraphics[width=3cm,height=1.3cm]{"{{info.imagesdir}}Ashoks_signature.png"}};
\end{tikzpicture}
\\
Ashok Kamath,\\
Chairman, Akshara Foundation\\
Bangalore\\
}

\pagebreak




\thispagestyle{plain}
{
\setlength{\parindent}{0in}
\begin{center}\large{\textbf{Ganitha Kalika Andolana  {{info.acadyear}} }} \\ \end{center}

~\\~\\

District Report of Gram Panchayat Level School Children Mathematics Competition \\
District : {{districtinfo.name}}  \\ [1ex]
Number of talukas participated in the competition: {{districtinfo.num_blocks}} \\[1ex]
Number of Gram Panchayats participating in the competition: {{districtinfo.num_gps}} \\


The following is a short report of the mathematics competition for 4th, 5th and 6th class school children held at Gram Panchayat level in the {{districtinfo.name}} district in collaboration with Gram Panchayat, Department of Public Education, Community and Akshara Foundation.\\
Mathematics competition was held in {{districtinfo.num_gps}} Gram Panchayats from {{districtinfo.num_blocks}} taluks in this district and {{districtinfo.totalstudents}} children of 4th, 5th and 6th class from {{districtinfo.num_schools}} government schools participated in the competition.\\[2ex]

\textbf{The overall results of the children who participated in the class-wise maths competition are as follows:}

\begin{longtable}{|P{3cm}|P{4cm}|P{2cm}|P{2cm}|P{2cm}|P{3cm}|} \hline
\textbf{Grades} & \textbf{Number of children participated} & \multicolumn{4}{|c|}{\textbf{Number of children scored (percentage)}}\\ \cline{3-6}
\rule{0cm}{0.3cm}& &  \textbf{<35\%} & \textbf{35\%-59\%}  & \textbf{60\%-74\%} & \textbf{75\%-100\%} \\ \hline \endhead

\rule{0cm}{0.3cm} {{ assessment["class"] }}th Grade & {{ assessment["overall_scores"]["total"]}} & {{assessment["overall_scores"]["below35"]}} & {{assessment["overall_scores"]["35to60"]}} & {{assessment["overall_scores"]["60to75"]}} & {{assessment["overall_scores"]["75to100"]}} \\ \hline

\end{longtable}

\begingroup
\fontsize{11}{12}
{
\setlength\LTleft{-0.1cm}
\setlength\LTright{-0.1cm}
\begin{longtable}{|P{2.5cm}|P{3.25cm}|c|c|c|c|c|c|} \hline
\textbf{Grades} & \textbf{Number of children participated} & \multicolumn{6}{|c|}{\textbf{Average percentage marks scored by children in basic operation}} \\ \cline{3-8}
\rule{0cm}{0.3cm} &&\textbf{\makecell{Number \\Sense}}&\textbf{\makecell{Place\\ Value}}&\textbf{Addition}&\textbf{Subtraction}&\textbf{Multiplication}&\textbf{Division} \\ \hline \endhead

\rule{0cm}{0.3cm} {{assessment["class"]}}th Grade & {{assessment["competency_scores"]["total"]}} & {{assessment["competency_scores"]["Number Recognition"]}}  & {{assessment["competency_scores"]["Place Value"]}} & {{assessment["competency_scores"]["Addition"]}} & {{assessment["competency_scores"]["Subtraction"]}} & {{assessment["competency_scores"]["Multiplication"]}} & {{assessment["competency_scores"]["Division"]}} \\ \hline

\end{longtable}
}
\endgroup


{
\centering{Result : Overall we can consider that the children studied in 5 year (6th class) school can do the following basic operations as percentage} \\[2ex]

\begingroup
\fontsize{8.75}{12}
{

{
\centering{
\begin{longtable}{|c|c|c|c|c|c|} \hline
\cellcolor{white}{\textbf{\makecell[b]{Children who can do \\number recognition correctly}} & \cellcolor{white}{\textbf{\makecell[b]{Children who can do \\place value correctly}} & \textbf{\makecell[b]{Children who can do \\addition correctly}} & \cellcolor{white}{\textbf{\makecell[b]{Children who can do \\subtraction correctly}} & \cellcolor{white}{\textbf{\makecell[b]{Children who can do \\multiplication correctly}} & \textbf{\makecell[b]{Children who can do \\division correctly}} \\ \hline \endhead

\rule{0cm}{0.3cm} {{percent_scores["assessments"][assessment]["Number Recognition"]}}&  {{percent_scores["assessments"][assessment]["Place Value"]}} & {{percent_scores["assessments"][assessment]["Addition"]}} & {{percent_scores["assessments"][assessment]["Subtraction"]}} & {{percent_scores["assessments"][assessment]["Multiplication"]}} &  {{percent_scores["assessments"][assessment]["Division"]}}  \\ \hline

\end{longtable}
}
}

{
\centering{
\begin{longtable}{|c|c|c|c|} \hline
 \textbf{\makecell[b]{Children who can do \\addition correctly}} & \textbf{\makecell[b]{Children who can do \\subtraction correctly}}& \textbf{\makecell[b]{Children who can do \\multiplication correctly}}& \textbf{\makecell[b]{Children who can do \\division correctly}} \\ \hline \endhead

\rule{0cm}{0.3cm} {{percent_scores["assessments"][assessment]["Addition"]}} & {{percent_scores["assessments"][assessment]["Subtraction"]}} & {{percent_scores["assessments"][assessment]["Multiplication"]}} &  {{percent_scores["assessments"][assessment]["Division"]}}  \\ \hline

\end{longtable}
}
}


}
\endgroup
}


\begin{flushleft}
To further improve the above results, we hope that Public Education Department, Gram Panchayat along with other stakeholders like Civic Amenities Committee, SDMC, parents, educational volunteers and school teachers will formulate a plan. \\[5ex]
\end{flushleft}

\centering{Thank You!} \\[3ex]

\begin{flushleft}
From\\
Akshara Foundation
\end{flushleft}
}
\end{document}

