\documentclass[10pt]{article}
\usepackage[a4paper,top=2.5cm,left=1cm,right=1cm,bottom=2cm,portrait]{geometry}
\usepackage{longtable}
\usepackage{graphicx}
\usepackage{makecell}
\usepackage{lastpage}
\usepackage{multirow}
\usepackage{xcolor}
\usepackage{colortbl}
\usepackage{fancyhdr}
\pagestyle{fancy}
\usepackage[T1]{fontenc}
\usepackage{textcomp}
\usepackage{fontenc}
\usepackage{xltxtra}


\newcommand{\kn}[1]{%Kannada text
{\fontspec{Kedage}%
#1
}}

%% Headers and footers

\rhead{}
\lhead{}


\setlength{\parindent}{0pt}

\setlength{\headheight}{2.7cm}
\renewcommand{\headrulewidth}{0pt}


\begin{titlepage}
    \begin{center}
        \vspace*{1cm}
        \textbf{MATH LEARNING LEVELS}
        \vspace{0.5cm}
        A SCHOOL LEVEL REPORT
        \vspace{1.5cm}
	    \textbf{ {{info.yeaar}} }
    \end{center}
\end{titlepage}

\begin{document}


\kn{ಗ್ರಾಮ ಪಂಚಾಯ್ತಿ, ಶಿಕ್ಷಣ ಇಲಾಖೆ, ಸಮುದಾಯ ಮತ್ತು ಅಕ್ಷರ ಫೌಂಡೇಶನ್ ಸಹಭಾಗಿತ್ವದಲ್ಲಿ ದಿನಾಂಕ} [% gp_info.date %] \kn{ರಂದು ನಡೆದ ಗ್ರಾಮ ಪಂಚಾಯ್ತಿ ಮಟ್ಟದಲ್ಲಿ 4, 5 ಮತ್ತು 6ನೇ ತರಗತಿಯ ಶಾಲಾ ಮಕ್ಕಳ ಗಣಿತ ಸ್ಪರ್ಧೆಯ ಕಿರು ವರದಿ ಈ ಕೆಳಕಂಡಂತಿದೆ.}\\[2ex]
\kn{ಈ ಗ್ರಾಮ ಪಂಚಾಯ್ತಿ ವ್ಯಾಪ್ತಿಯಲ್ಲಿ} [% gp_info.school_count %] \kn{ಸರ್ಕಾರಿ ಶಾಲೆಗಳಿಂದ 4, 5 ಮತ್ತು 6ನೇ ತರಗತಿಯ} [% gp_info.total %] \kn{ಮಕ್ಕಳು ಗ್ರಾಮ ಪಂಚಾಯ್ತಿ ಮಟ್ಟದ ಗಣಿತ ಸ್ಪರ್ಧೆಯಲ್ಲಿ ಭಾಗವಹಿಸಿದ್ದರು.}\\[2ex]

\textbf{\kn{ತರಗತಿವಾರು ಗಣಿತ ಸ್ಪರ್ಧೆಯಲ್ಲಿ ಭಾಗವಹಿಸಿದ್ದ ಮಕ್ಕಳ ಒಟ್ಟಾರೆ ಫಲಿತಾಂಶಗಳು ಈ ಕೆಳಗಿನಂತೆ ಇವೆ.}}

\begin{longtable}{|p{2cm}|c|c|c|c|c|} \hline

\textbf{\kn{ತರಗತಿ}} & \textbf{\kn{ಭಾಗವಹಿಸಿದ ಮಕ್ಕಳ ಸಂಖ್ಯೆ}} & \textbf{\makecell[b]{35\% \kn{ಕ್ಕಿಂತ ಕಡಿಮೆ}\\ \kn{ಅಂಕ ಪಡೆದವರ ಸಂಖ್ಯೆ}}} & \textbf{\makecell[b]{36-60\% \\ \kn{ಅಂಕ ಪಡೆದವರ ಸಂಖ್ಯೆ}}} & \textbf{\makecell[b]{61-75\% \\ \kn{ಅಂಕ ಪಡೆದವರ ಸಂಖ್ಯೆ}}} & \textbf{\makecell[b]{76-100\% \\ \kn{ಅಂಕ ಪಡೆದವರ ಸಂಖ್ಯೆ}}} \\ \hline
\rule{0cm}{0.75cm} 4\kn{ನೇ ತರಗತಿ} & [% class_4.tb1.total %] & [% class_4.tb1.a %] & [% class_4.tb1.b %] & [% class_4.tb1.c %] & [% class_4.tb1.d %] \\ \hline
\rule{0cm}{0.75cm} 5\kn{ನೇ ತರಗತಿ} & [% class_5.tb1.total %] & [% class_5.tb1.a %] & [% class_5.tb1.b %] & [% class_5.tb1.c %] & [% class_5.tb1.d %] \\ \hline
\rule{0cm}{0.75cm} 6\kn{ನೇ ತರಗತಿ} & [% class_6.tb1.total %] & [% class_6.tb1.a %] & [% class_6.tb1.b %] & [% class_6.tb1.c %] & [% class_6.tb1.d %] \\ \hline

\end{longtable}

%%\textbf{\kn{ಗಣಿತ ಮೂಲಭೂತ ಸಾಮರ್ಥ್ಯದ ಮಕ್ಕಳ ಫಲಿತಾಂಶ}}

%%\begin{longtable}{|p{2cm}|p{2.7cm}|p{3cm}|p{3cm}|p{3cm}|p{3cm}|} \hline
\begin{longtable}{@{\extracolsep{\fill}}|p{2cm}|c|c|c|c|c|@{}} \hline

\textbf{\kn{ತರಗತಿ}} & \textbf{\kn{ಒಟ್ಟು ಮಕ್ಕಳು}} \rule{1.6cm}{0cm} & \textbf{\makecell[b]{\kn{ಕೂಡುವ ಲೆಕ್ಕವನ್ನು}\\ \kn{ಸರಿಯಾಗಿ ಮಾಡಿದವರ}\\ \kn{ಮಕ್ಕಳ ಸಂಖ್ಯೆ}}} & \textbf{\makecell[b]{\kn{ಕಳೆಯುವ ಲೆಕ್ಕವನ್ನು}\\ \kn{ಸರಿಯಾಗಿ ಮಾಡಿದವರ}\\ \kn{ಮಕ್ಕಳ ಸಂಖ್ಯೆ}}} & \textbf{\makecell[b]{\kn{ಗುಣಿಸುವ ಲೆಕ್ಕವನ್ನು}\\ \kn{ಸರಿಯಾಗಿ ಮಾಡಿದವರ}\\ \kn{ಮಕ್ಕಳ ಸಂಖ್ಯೆ}}} & \textbf{\makecell[b]{\kn{ಭಾಗಾಕಾರ ಲೆಕ್ಕವನ್ನು}\\ \kn{ಸರಿಯಾಗಿ ಮಾಡಿದವರ}\\ \kn{ಮಕ್ಕಳ ಸಂಖ್ಯೆ}}} \\ \hline
%%\endhead 

\rule{0cm}{0.75cm} 4\kn{ನೇ ತರಗತಿ} & [% class_4.tb2.total %] & [% class_4.tb2.a %] & [% class_4.tb2.b %] & [% class_4.tb2.c %] & [% class_4.tb2.d %] \\ \hline
\rule{0cm}{0.75cm} 5\kn{ನೇ ತರಗತಿ} & [% class_5.tb2.total %] & [% class_5.tb2.a %] & [% class_5.tb2.b %] & [% class_5.tb2.c %] & [% class_5.tb2.d %] \\ \hline
\rule{0cm}{0.75cm} 6\kn{ನೇ ತರಗತಿ} & [% class_6.tb2.total %] & [% class_6.tb2.a %] & [% class_6.tb2.b %] & [% class_6.tb2.c %] & [% class_6.tb2.d %] \\ \hline

\end{longtable}

\kn{ಈ ವರದಿಯಿಂದ ತಿಳಿಯುವುದೇನೆಂದರೆ,} \kn{5 ವರ್ಷ ಶಾಲಾ ದಿನಗಳನ್ನು ಮುಗಿಸಿದ} [% class_6.tb3.total %] \kn{ಮಕ್ಕಳಲ್ಲಿ,}\\[2ex]
[% class_6.tb3.a %]\% \kn{ಮಕ್ಕಳಿಗೆ ಕೂಡುವ ಲೆಕ್ಕ ಬರುತ್ತದೆ.} \\ [2ex]
[% class_6.tb3.b %]\% \kn{ಮಕ್ಕಳಿಗೆ ಕಳೆಯುವ ಲೆಕ್ಕ ಬರುತ್ತದೆ.} \\ [2ex]
[% class_6.tb3.c %]\% \kn{ಮಕ್ಕಳಿಗೆ ಗುಣಿಸುವ ಲೆಕ್ಕ ಬರುತ್ತದೆ.} \\ [2ex]
[% class_6.tb3.d %]\% \kn{ಮಕ್ಕಳಿಗೆ ಭಾಗಾಕಾರ ಲೆಕ್ಕ ಬರುತ್ತದೆ.} \\ [2ex]

\kn{ಈ ಮೇಲ್ಕಂಡ ಫಲಿತಾಂಶಗಳು ಇನ್ನೂ ಉತ್ತಮ ಪಡಿಸಲು ಗ್ರಾಮ ಪಂಚಾಯ್ತಿ ಜೊತೆಗೆ ಇತರೆ ಭಾಗೀದಾರರಾದ ನಾಗರೀಕ ಸೌಕರ್ಯ ಸಮಿತಿ}(Civic Amenities Committee), SDMC, \kn{ಪೋಷಕರು, ಶೈಕ್ಷಣಿಕ ಸ್ವಯಂಸೇವಕರು ಮತ್ತು ಶಾಲಾ ಶಿಕ್ಷಕರ ಜೊತೆಗೂಡಿ ಯೋಜನೆಯನ್ನು ರೂಪಿಸುತ್ತೀರಾ ಎಂದು ಭಾವಿಸುತ್ತೇವೆ.}\\[3ex]

\center{\kn{ವಂದನೆಗಳೊಂದಿಗೆ}}\\[3ex]

%%\kn{ಅಶೋಕ್ ಕಾಮತ್} \hfill \kn{ಕೆ. ವೈಜಯಂತಿ} \rule{3.68cm}{0cm}\\                                                                    
%%\kn{ಅಧ್ಯಕ್ಷರು, ಅಕ್ಷರ ಫೌಂಡೇಶನ್} \hfill \kn{ಮುಖ್ಯಸ್ಥರು, ಸಂಪನ್ಮೂಲ ಮತ್ತು ಸಂಶೋಧನೆ}\rule{0.1cm}{0cm}\\
%%\kn{ಬೆಂಗಳೂರು.} \hfill \kn{ಅಕ್ಷರ ಫೌಂಡೇಶನ್, ಬೆಂಗಳೂರು.}\rule{1.4cm}{0cm}
\kn{ಜೆ. ವಿ. ಶಂಕರನಾರಾಯಣ} \hfill \kn{ನಾಗರಾಜ್ ಪ್ರಭು} \rule{3.5cm}{0cm}\\                                                                    
\kn{ಮುಖ್ಯಸ್ಥರು, ಕಾರ್ಯಾಚರಣೆ ಮತ್ತು ಸಮುದಾಯ ಅಭಿವೃದ್ದಿ} \hfill \kn{ಮುಖ್ಯಸ್ಥರು, ಗಣಿತ ಸಂಪನ್ಮೂಲ ಮತ್ತು}GKA\rule{0.17cm}{0cm}\\
\kn{ಅಕ್ಷರ ಫೌಂಡೇಶನ್, ಬೆಂಗಳೂರು.} \hfill \kn{ಅಕ್ಷರ ಫೌಂಡೇಶನ್, ಬೆಂಗಳೂರು.}\rule{1.5cm}{0cm}
\end{document}
